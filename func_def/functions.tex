
\subsection[\texorpdfstring{\protect\(F_1\protect\): Sphere/Sphere}{F1: Sphere/Sphere}]{\texorpdfstring{\protect\(F_1\protect\): Sphere/Sphere}{}}
\label{index:f1}\label{index:sphere-sphere}
Combination of two sphere functions (\href{https://coco.gforge.inria.fr/downloads/download16.00/bbobdocfunctions.pdf\#page=5}{\(f_1\) in the \bbob suite}).

\noindent Both objectives are unimodal, highly symmetric, rotational and scale
invariant. The Pareto set is known to be a straight line and the Pareto
front is convex. Furthermore, the normalized hypervolume value of the
entire Pareto front with respect to the nadir point as reference point
can be computed analytically as the integral
\(1-\int_{0}^{1} (1-\sqrt{x})^2dx = -\frac{1}{2}+\frac{4}{3}=0.833333\ldots\).

Considered as the simplest bi-objective problem in
continuous domain.
%
Contained in the \emph{separable - separable} function group.

% #1: function string (2 digits)
% #2: dimension string (2 digits)
% #3: instance string (2 digits)
% #4: function number
% #5: dimension
\bbobfigs{01}{05}{01}{1}{5}


\subsection[\texorpdfstring{\protect\(F_2\protect\): Sphere/Ellipsoid separable}{F2: Sphere/Ellipsoid separable}]{\texorpdfstring{\protect\(F_2\protect\): Sphere/Ellipsoid separable}{}}
\label{index:f2}\label{index:sphere-ellipsoid-separable}
Combination of the sphere function (\href{https://coco.gforge.inria.fr/downloads/download16.00/bbobdocfunctions.pdf\#page=5}{\(f_1\) in the \bbob suite})
and the separable ellipsoid function (\href{https://coco.gforge.inria.fr/downloads/download16.00/bbobdocfunctions.pdf\#page=10}{\(f_2\) in the \bbob suite}).

Both objectives are unimodal and separable. While the first objective is
truly convex-quadratic with a condition number of 1, the second
objective is only globally quadratic with smooth local
irregularities and highly ill-conditioned with a condition number of
about \(10^6\).

Contained in the \emph{separable - separable} function group.

% #1: function string (2 digits)
% #2: dimension string (2 digits)
% #3: instance string (2 digits)
% #4: function number
% #5: dimension
\bbobfigs{02}{05}{01}{2}{5}

\subsection[\texorpdfstring{\protect\(F_3\protect\): Sphere/Attractive sector}{F3: Sphere/Attractive sector}]{\texorpdfstring{\protect\(F_3\protect\): Sphere/Attractive sector}{}}
\label{index:f3}\label{index:sphere-attractive-sector}
Combination of the sphere function (\href{https://coco.gforge.inria.fr/downloads/download16.00/bbobdocfunctions.pdf\#page=5}{\(f_1\) in the \bbob suite})
and the attractive sector function (\href{https://coco.gforge.inria.fr/downloads/download16.00/bbobdocfunctions.pdf\#page=30}{\(f_6\) in the \bbob suite}).

Both objective functions are unimodal, but only the first objective is
separable and truly convex quadratic. The attractive sector
function is highly asymmetric, where only one \emph{hypercone} (with
angular base area) with a volume of roughly \((1/2)^n\)
yields low function values. The optimum of it is located at the tip
of this cone.

Contained in the \emph{separable - moderate} function group.

% #1: function string (2 digits)
% #2: dimension string (2 digits)
% #3: instance string (2 digits)
% #4: function number
% #5: dimension
\bbobfigs{03}{05}{01}{3}{5}


\subsection[\texorpdfstring{\protect\(F_4\protect\): Sphere/Rosenbrock original}{F4: Sphere/Rosenbrock original}]{\texorpdfstring{\protect\(F_4\protect\): Sphere/Rosenbrock original}{}}
\label{index:sphere-rosenbrock-original}\label{index:f4}
Combination of the sphere function (\href{https://coco.gforge.inria.fr/downloads/download16.00/bbobdocfunctions.pdf\#page=5}{\(f_1\) in the \bbob suite})
and the original, i.e., unrotated Rosenbrock function (\href{https://coco.gforge.inria.fr/downloads/download16.00/bbobdocfunctions.pdf\#page=40}{\(f_8\) in the \bbob suite}).

The first objective is separable and truly convex, the second
objective is partially separable (tri-band structure). The first
objective is unimodal while the second objective has a local
optimum with an attraction volume of about 25\%.

Contained in the \emph{separable - moderate} function group.

% #1: function string (2 digits)
% #2: dimension string (2 digits)
% #3: instance string (2 digits)
% #4: function number
% #5: dimension
\bbobfigs{04}{05}{01}{4}{5}


\subsection[\texorpdfstring{\protect\(F_5\protect\): Sphere/Sharp ridge}{F5: Sphere/Sharp ridge}]{\texorpdfstring{\protect\(F_5\protect\): Sphere/Sharp ridge}{}}
\label{index:sphere-sharp-ridge}\label{index:f5}
Combination of the sphere function (\href{https://coco.gforge.inria.fr/downloads/download16.00/bbobdocfunctions.pdf\#page=5}{\(f_1\) in the \bbob suite})
and the sharp ridge function (\href{https://coco.gforge.inria.fr/downloads/download16.00/bbobdocfunctions.pdf\#page=65}{\(f_{13}\) in the \bbob suite}).

Both objective functions are unimodal.
In addition to the simple, separable, and differentiable first
objective, a sharp, i.e., non-differentiable ridge has to be
followed for optimizing the (non-separable) second objective. The
gradient towards the ridge remains constant, when the ridge is
approached from a given point.
Approaching the ridge is initially effective, but becomes ineffective
close to the ridge when the ridge needs to be followed in direction
to its optimum.  The necessary change in \emph{search behavior} close to
the ridge is difficult to diagnose, because the gradient
towards the ridge does not flatten out.

Contained in the \emph{separable - ill-conditioned} function group.

% #1: function string (2 digits)
% #2: dimension string (2 digits)
% #3: instance string (2 digits)
% #4: function number
% #5: dimension
\bbobfigs{05}{05}{01}{5}{5}


\subsection[\texorpdfstring{\protect\(F_6\protect\): Sphere/Sum of Different Powers}{F6: Sphere/Sum of Different Powers}]{\texorpdfstring{\protect\(F_6\protect\): Sphere/Sum of Different Powers}{}}
\label{index:sphere-sum-of-different-powers}\label{index:f6}
Combination of the sphere function (\href{https://coco.gforge.inria.fr/downloads/download16.00/bbobdocfunctions.pdf\#page=5}{\(f_1\) in the \bbob suite})
and the sum of different powers function (\href{https://coco.gforge.inria.fr/downloads/download16.00/bbobdocfunctions.pdf\#page=70}{\(f_{14}\) in the \bbob suite}).

Both objective functions are unimodal. The first objective is
separable, the second non-separable.
When approaching the second objective's optimum, the difference
in sensitivity between different directions in search space
increases unboundedly.

Contained in the \emph{separable - ill-conditioned} function group.

% #1: function string (2 digits)
% #2: dimension string (2 digits)
% #3: instance string (2 digits)
% #4: function number
% #5: dimension
\bbobfigs{06}{05}{01}{6}{5}


\subsection[\texorpdfstring{\protect\(F_7\protect\): Sphere/Rastrigin}{F7: Sphere/Rastrigin}]{\texorpdfstring{\protect\(F_7\protect\): Sphere/Rastrigin}{}}
\label{index:sphere-rastrigin}\label{index:f7}
Combination of the sphere function (\href{https://coco.gforge.inria.fr/downloads/download16.00/bbobdocfunctions.pdf\#page=5}{\(f_1\) in the \bbob suite})
and the Rastrigin function (\href{https://coco.gforge.inria.fr/downloads/download16.00/bbobdocfunctions.pdf\#page=75}{\(f_{15}\) in the \bbob suite}).

In addition to the simple sphere function, the prototypical highly
multimodal Rastrigin function needs to be solved which has originally
a very regular and symmetric structure for the placement of the optima.
Here, however, transformations are performed to alleviate
the original symmetry and regularity in the second objective.

The properties of the second objective contain non-separability,
multimodality (roughly \(10^n\) local optima), a conditioning of
about 10, and a large global amplitude compared to the local amplitudes.

Contained in the \emph{separable - multimodal} function group.

% #1: function string (2 digits)
% #2: dimension string (2 digits)
% #3: instance string (2 digits)
% #4: function number
% #5: dimension
\bbobfigs{07}{05}{01}{7}{5}


\subsection[\texorpdfstring{\protect\(F_8\protect\): Sphere/Schaffer F7, condition 10}{F8: Sphere/Schaffer F7, condition 10}]{\texorpdfstring{\protect\(F_8\protect\): Sphere/Schaffer F7, condition 10}{}}
\label{index:f8}\label{index:sphere-schaffer-f7-condition-10}
Combination of the sphere function (\href{https://coco.gforge.inria.fr/downloads/download16.00/bbobdocfunctions.pdf\#page=5}{\(f_1\) in the \bbob suite})
and the Schaffer F7 function with condition number 10 (\href{https://coco.gforge.inria.fr/downloads/download16.00/bbobdocfunctions.pdf\#page=85}{\(f_{17}\) in the \bbob suite}).

In addition to the simple sphere function, an asymmetric, non-separable,
and highly multimodal function needs to be solved to approach the Pareto
front/Pareto set where the frequency and amplitude of the modulation
in the second objective vary. The conditioning of the second objective
and thus the entire bi-objective function is low.

Contained in the \emph{separable - multimodal} function group.

% #1: function string (2 digits)
% #2: dimension string (2 digits)
% #3: instance string (2 digits)
% #4: function number
% #5: dimension
\bbobfigs{08}{05}{01}{8}{5}


\subsection[\texorpdfstring{\protect\(F_9\protect\): Sphere/Schwefel x*sin(x)}{F9: Sphere/Schwefel x*sin(x)}]{\texorpdfstring{\protect\(F_9\protect\): Sphere/Schwefel x*sin(x)}{}}
\label{index:f9}\label{index:sphere-schwefel-x-sin-x}
Combination of the sphere function (\href{https://coco.gforge.inria.fr/downloads/download16.00/bbobdocfunctions.pdf\#page=5}{\(f_1\) in the \bbob suite})
and the Schwefel function (\href{https://coco.gforge.inria.fr/downloads/download16.00/bbobdocfunctions.pdf\#page=100}{\(f_{20}\) in the \bbob suite}).

While the first objective function is separable and unimodal,
the second objective function is partially separable and highly
multimodal---having the most prominent \(2^n\) minima located
comparatively close to the corners of the unpenalized search area.

Contained in the \emph{separable - weakly-structured} function group.

% #1: function string (2 digits)
% #2: dimension string (2 digits)
% #3: instance string (2 digits)
% #4: function number
% #5: dimension
\bbobfigs{09}{05}{01}{9}{5}


\subsection[\texorpdfstring{\protect\(F_{10}\protect\): Sphere/Gallagher 101 peaks}{F10: Sphere/Gallagher 101 peaks}]{\texorpdfstring{\protect\(F_{10}\protect\): Sphere/Gallagher 101 peaks}{}}
\label{index:f10}\label{index:sphere-gallagher-101-peaks}
Combination of the sphere function (\href{https://coco.gforge.inria.fr/downloads/download16.00/bbobdocfunctions.pdf\#page=5}{\(f_1\) in the \bbob suite})
and the Gallagher function with 101 peaks (\href{https://coco.gforge.inria.fr/downloads/download16.00/bbobdocfunctions.pdf\#page=105}{\(f_{21}\) in the \bbob suite}).

While the first objective function is separable and unimodal,
the second objective function is non-separable and consists
of 101 optima with position and height being unrelated and
randomly chosen (different for each instantiation of the function).
The conditioning around the global optimum of the second
objective function is about 30.

Contained in the \emph{separable - weakly-structured} function group.

% #1: function string (2 digits)
% #2: dimension string (2 digits)
% #3: instance string (2 digits)
% #4: function number
% #5: dimension
\bbobfigs{10}{05}{01}{10}{5}


\subsection[\texorpdfstring{\protect\(F_{11}\protect\): Ellipsoid separable/Ellipsoid separable}{F11: Ellipsoid separable/Ellipsoid separable}]{\texorpdfstring{\protect\(F_{11}\protect\): Ellipsoid separable/Ellipsoid separable}{}}
\label{index:ellipsoid-separable-ellipsoid-separable}\label{index:f11}
Combination of two separable ellipsoid functions (\href{https://coco.gforge.inria.fr/downloads/download16.00/bbobdocfunctions.pdf\#page=10}{\(f_2\) in the \bbob suite}).

Both objectives are unimodal, separable, only globally
quadratic with smooth local irregularities, and highly
ill-conditioned with a condition number of
about \(10^6\).

Contained in the \emph{separable - separable} function group.

% #1: function string (2 digits)
% #2: dimension string (2 digits)
% #3: instance string (2 digits)
% #4: function number
% #5: dimension
\bbobfigs{11}{05}{01}{11}{5}


\subsection[\texorpdfstring{\protect\(F_{12}\protect\): Ellipsoid separable/Attractive sector}{F12: Ellipsoid separable/Attractive sector}]{\texorpdfstring{\protect\(F_{12}\protect\): Ellipsoid separable/Attractive sector}{}}
\label{index:f12}\label{index:ellipsoid-separable-attractive-sector}
Combination of the separable ellipsoid function (\href{https://coco.gforge.inria.fr/downloads/download16.00/bbobdocfunctions.pdf\#page=10}{\(f_2\) in the \bbob suite})
and the attractive sector function (\href{https://coco.gforge.inria.fr/downloads/download16.00/bbobdocfunctions.pdf\#page=30}{\(f_6\) in the \bbob suite}).

Both objective functions are unimodal but only the first
one is separable. The first objective function, in addition,
is globally quadratic with smooth local irregularities, and
highly ill-conditioned with a condition number of about
\(10^6\). The second objective function is highly
asymmetric, where only one \emph{hypercone} (with
angular base area) with a volume of roughly \((1/2)^n\)
yields low function values. The optimum of it is located at
the tip of this cone.

Contained in the \emph{separable - moderate} function group.

% #1: function string (2 digits)
% #2: dimension string (2 digits)
% #3: instance string (2 digits)
% #4: function number
% #5: dimension
\bbobfigs{12}{05}{01}{12}{5}


\subsection[\texorpdfstring{\protect\(F_{13}\protect\): Ellipsoid separable/Rosenbrock original}{F13: Ellipsoid separable/Rosenbrock original}]{\texorpdfstring{\protect\(F_{13}\protect\): Ellipsoid separable/Rosenbrock original}{}}
\label{index:f13}\label{index:ellipsoid-separable-rosenbrock-original}
Combination of the separable ellipsoid function (\href{https://coco.gforge.inria.fr/downloads/download16.00/bbobdocfunctions.pdf\#page=10}{\(f_2\) in the \bbob suite}) and the original, i.e., unrotated Rosenbrock function
(\href{https://coco.gforge.inria.fr/downloads/download16.00/bbobdocfunctions.pdf\#page=40}{\(f_8\) in the \bbob suite}).

Only the first objective is separable and unimodal. The second
objective is partially separable (tri-band structure) and has a local
optimum with an attraction volume of about 25\%.
In addition, the first objective function shows smooth local
irregularities from a globally convex quadratic function and is
highly ill-conditioned with a condition number of about
\(10^6\).

Contained in the \emph{separable - moderate} function group.

% #1: function string (2 digits)
% #2: dimension string (2 digits)
% #3: instance string (2 digits)
% #4: function number
% #5: dimension
\bbobfigs{13}{05}{01}{13}{5}


\subsection[\texorpdfstring{\protect\(F_{14}\protect\): Ellipsoid separable/Sharp ridge}{F14: Ellipsoid separable/Sharp ridge}]{\texorpdfstring{\protect\(F_{14}\protect\): Ellipsoid separable/Sharp ridge}{}}
\label{index:f14}\label{index:ellipsoid-separable-sharp-ridge}
Combination of the separable ellipsoid function (\href{https://coco.gforge.inria.fr/downloads/download16.00/bbobdocfunctions.pdf\#page=10}{\(f_2\) in the \bbob suite}) and the sharp ridge function (\href{https://coco.gforge.inria.fr/downloads/download16.00/bbobdocfunctions.pdf\#page=65}{\(f_{13}\) in the \bbob suite}).

Both objective functions are unimodal but only the first one is
separable.

The first objective is globally quadratic but with smooth local
irregularities and highly ill-conditioned with a condition number of
about \(10^6\). For optimizing the second objective, a sharp,
i.e., non-differentiable ridge has to be followed.

Contained in the \emph{separable - ill-conditioned} function group.

% #1: function string (2 digits)
% #2: dimension string (2 digits)
% #3: instance string (2 digits)
% #4: function number
% #5: dimension
\bbobfigs{14}{05}{01}{14}{5}


\subsection[\texorpdfstring{\protect\(F_{15}\protect\): Ellipsoid separable/Sum of Different Powers}{F15: Ellipsoid separable/Sum of Different Powers}]{\texorpdfstring{\protect\(F_{15}\protect\): Ellipsoid separable/Sum of Different Powers}{}}
\label{index:ellipsoid-separable-sum-of-different-powers}\label{index:f15}
Combination of the separable ellipsoid function (\href{https://coco.gforge.inria.fr/downloads/download16.00/bbobdocfunctions.pdf\#page=10}{\(f_2\) in the \bbob suite}) and the sum of different powers function
(\href{https://coco.gforge.inria.fr/downloads/download16.00/bbobdocfunctions.pdf\#page=70}{\(f_{14}\) in the \bbob suite}).

Both objective functions are unimodal but only the first one is
separable.

The first objective is globally quadratic but with smooth local
irregularities and highly ill-conditioned with a condition number of
about \(10^6\). When approaching the second objective's optimum,
the sensitivities of the variables in the rotated search space become
more and more different.

Contained in the \emph{separable - ill-conditioned} function group.

% #1: function string (2 digits)
% #2: dimension string (2 digits)
% #3: instance string (2 digits)
% #4: function number
% #5: dimension
\bbobfigs{15}{05}{01}{15}{5}


\subsection[\texorpdfstring{\protect\(F_{16}\protect\): Ellipsoid separable/Rastrigin}{F16: Ellipsoid separable/Rastrigin}]{\texorpdfstring{\protect\(F_{16}\protect\): Ellipsoid separable/Rastrigin}{}}
\label{index:ellipsoid-separable-rastrigin}\label{index:f16}
Combination of the separable ellipsoid function (\href{https://coco.gforge.inria.fr/downloads/download16.00/bbobdocfunctions.pdf\#page=10}{\(f_2\) in the \bbob suite}) and the Rastrigin function (\href{https://coco.gforge.inria.fr/downloads/download16.00/bbobdocfunctions.pdf\#page=75}{\(f_{15}\) in the \bbob suite}).

The objective functions show rather opposite properties.
The first one is separable, the second not. The first one
is unimodal, the second highly multimodal (roughly \(10^n\) local
optima). The first one is highly ill-conditioning (condition number of
\(10^6\)), the second one has a conditioning of about 10. Local
non-linear transformations are performed in both objective functions
to alleviate the original symmetry and regularity of the two
baseline functions.

Contained in the \emph{separable - multimodal} function group.

% #1: function string (2 digits)
% #2: dimension string (2 digits)
% #3: instance string (2 digits)
% #4: function number
% #5: dimension
\bbobfigs{16}{05}{01}{16}{5}


\subsection[\texorpdfstring{\protect\(F_{17}\protect\): Ellipsoid separable/Schaffer F7, condition 10}{F17: Ellipsoid separable/Schaffer F7, condition 10}]{\texorpdfstring{\protect\(F_{17}\protect\): Ellipsoid separable/Schaffer F7, condition 10}{}}
\label{index:ellipsoid-separable-schaffer-f7-condition-10}\label{index:f17}
Combination of the separable ellipsoid function (\href{https://coco.gforge.inria.fr/downloads/download16.00/bbobdocfunctions.pdf\#page=10}{\(f_2\) in the \bbob suite}) and the Schaffer F7 function with condition number 10
(\href{https://coco.gforge.inria.fr/downloads/download16.00/bbobdocfunctions.pdf\#page=85}{\(f_{17}\) in the \bbob suite}).

Also here, both single objectives possess opposing properties.
The first objective is unimodal, besides small local non-linearities symmetric,
separable and highly ill-conditioned while the second objective is highly
multimodal, asymmetric, and non-separable, with only a low conditioning.

Contained in the \emph{separable - multimodal} function group.

% #1: function string (2 digits)
% #2: dimension string (2 digits)
% #3: instance string (2 digits)
% #4: function number
% #5: dimension
\bbobfigs{17}{05}{01}{17}{5}


\subsection[\texorpdfstring{\protect\(F_{18}\protect\): Ellipsoid separable/Schwefel x*sin(x)}{F18: Ellipsoid separable/Schwefel x*sin(x)}]{\texorpdfstring{\protect\(F_{18}\protect\): Ellipsoid separable/Schwefel x*sin(x)}{}}
\label{index:ellipsoid-separable-schwefel-x-sin-x}\label{index:f18}
Combination of the separable ellipsoid function (\href{https://coco.gforge.inria.fr/downloads/download16.00/bbobdocfunctions.pdf\#page=10}{\(f_2\) in the \bbob suite}) and the Schwefel function (\href{https://coco.gforge.inria.fr/downloads/download16.00/bbobdocfunctions.pdf\#page=100}{\(f_{20}\) in the \bbob suite}).

The first objective is unimodal, separable and highly ill-conditioned.
The second objective is partially separable and highly multimodal---having
the most prominent \(2^n\) minima located comparatively close to the
corners of the unpenalized search area.

Contained in the \emph{separable - weakly-structured} function group.

% #1: function string (2 digits)
% #2: dimension string (2 digits)
% #3: instance string (2 digits)
% #4: function number
% #5: dimension
\bbobfigs{18}{05}{01}{18}{5}


\subsection[\texorpdfstring{\protect\(F_{19}\protect\): Ellipsoid separable/Gallagher 101 peaks}{F19: Ellipsoid separable/Gallagher 101 peaks}]{\texorpdfstring{\protect\(F_{19}\protect\): Ellipsoid separable/Gallagher 101 peaks}{}}
\label{index:ellipsoid-separable-gallagher-101-peaks}\label{index:f19}
Combination of the separable ellipsoid function (\href{https://coco.gforge.inria.fr/downloads/download16.00/bbobdocfunctions.pdf\#page=10}{\(f_2\) in the \bbob suite}) and the Gallagher function with 101 peaks (\href{https://coco.gforge.inria.fr/downloads/download16.00/bbobdocfunctions.pdf\#page=105}{\(f_{21}\) in the \bbob suite}).

While the first objective function is separable, unimodal, and
highly ill-conditioned (condition number of about \(10^6\)),
the second objective function is non-separable and consists
of 101 optima with position and height being unrelated and
randomly chosen (different for each instantiation of the function).
The conditioning around the global optimum of the second
objective function is about 30.

Contained in the \emph{separable - weakly-structured} function group.

% #1: function string (2 digits)
% #2: dimension string (2 digits)
% #3: instance string (2 digits)
% #4: function number
% #5: dimension
\bbobfigs{19}{05}{01}{19}{5}


\subsection[\texorpdfstring{\protect\(F_{20}\protect\): Attractive sector/Attractive sector}{F20: Attractive sector/Attractive sector}]{\texorpdfstring{\protect\(F_{20}\protect\): Attractive sector/Attractive sector}{}}
\label{index:f20}\label{index:attractive-sector-attractive-sector}
Combination of two attractive sector functions (\href{https://coco.gforge.inria.fr/downloads/download16.00/bbobdocfunctions.pdf\#page=30}{\(f_6\) in the \bbob suite}).
Both functions are unimodal and highly asymmetric, where only one
\emph{hypercone} (with angular base area) per objective with a volume of
roughly \((1/2)^n\) yields low function values. The objective
functions' optima are located at the tips of those two cones.

Contained in the \emph{moderate - moderate} function group.

% #1: function string (2 digits)
% #2: dimension string (2 digits)
% #3: instance string (2 digits)
% #4: function number
% #5: dimension
\bbobfigs{20}{05}{01}{20}{5}


\subsection[\texorpdfstring{\protect\(F_{21}\protect\): Attractive sector/Rosenbrock original}{F21: Attractive sector/Rosenbrock original}]{\texorpdfstring{\protect\(F_{21}\protect\): Attractive sector/Rosenbrock original}{}}
\label{index:f21}\label{index:attractive-sector-rosenbrock-original}
Combination of the attractive sector function (\href{https://coco.gforge.inria.fr/downloads/download16.00/bbobdocfunctions.pdf\#page=30}{\(f_6\) in the \bbob suite}) and the Rosenbrock function (\href{https://coco.gforge.inria.fr/downloads/download16.00/bbobdocfunctions.pdf\#page=40}{\(f_8\) in the \bbob suite}).

The first function is unimodal but highly asymmetric, where only one
\emph{hypercone} (with angular base area) with a volume of
roughly \((1/2)^n\) yields low function values (with the
optimum at the tip of the cone). The second
objective is partially separable (tri-band structure) and has a local
optimum with an attraction volume of about 25\%.

Contained in the \emph{moderate - moderate} function group.

% #1: function string (2 digits)
% #2: dimension string (2 digits)
% #3: instance string (2 digits)
% #4: function number
% #5: dimension
\bbobfigs{21}{05}{01}{21}{5}


\subsection[\texorpdfstring{\protect\(F_{22}\protect\): Attractive sector/Sharp ridge}{F22: Attractive sector/Sharp ridge}]{\texorpdfstring{\protect\(F_{22}\protect\): Attractive sector/Sharp ridge}{}}
\label{index:f22}\label{index:attractive-sector-sharp-ridge}
Combination of the attractive sector function (\href{https://coco.gforge.inria.fr/downloads/download16.00/bbobdocfunctions.pdf\#page=30}{\(f_6\) in the \bbob suite}) and the sharp ridge function (\href{https://coco.gforge.inria.fr/downloads/download16.00/bbobdocfunctions.pdf\#page=65}{\(f_{13}\) in the \bbob suite}).

Both objective functions are unimodal and non-separable. The
first objective is highly asymmetric in the sense that only one
\emph{hypercone} (with angular base area) with a volume of
roughly \((1/2)^n\) yields low function values (with the
optimum at the tip of the cone). For optimizing the second
objective, a sharp, i.e., non-differentiable ridge has to be followed.

Contained in the \emph{moderate - ill-conditioned} function group.

% #1: function string (2 digits)
% #2: dimension string (2 digits)
% #3: instance string (2 digits)
% #4: function number
% #5: dimension
\bbobfigs{22}{05}{01}{22}{5}


\subsection[\texorpdfstring{\protect\(F_{23}\protect\): Attractive sector/Sum of Different Powers}{F23: Attractive sector/Sum of Different Powers}]{\texorpdfstring{\protect\(F_{23}\protect\): Attractive sector/Sum of Different Powers}{}}
\label{index:f23}\label{index:attractive-sector-sum-of-different-powers}
Combination of the attractive sector function (\href{https://coco.gforge.inria.fr/downloads/download16.00/bbobdocfunctions.pdf\#page=30}{\(f_6\) in the \bbob suite}) and the sum of different powers function
(\href{https://coco.gforge.inria.fr/downloads/download16.00/bbobdocfunctions.pdf\#page=70}{\(f_{14}\) in the \bbob suite}).

Both objective functions are unimodal and non-separable. The
first objective is highly asymmetric in the sense that only one
\emph{hypercone} (with angular base area) with a volume of
roughly \((1/2)^n\) yields low function values (with the
optimum at the tip of the cone). When approaching the second
objective's optimum, the sensitivities of the variables in the
rotated search space become more and more different.

Contained in the \emph{moderate - ill-conditioned} function group.

% #1: function string (2 digits)
% #2: dimension string (2 digits)
% #3: instance string (2 digits)
% #4: function number
% #5: dimension
\bbobfigs{23}{05}{01}{23}{5}


\subsection[\texorpdfstring{\protect\(F_{24}\protect\): Attractive sector/Rastrigin}{F24: Attractive sector/Rastrigin}]{\texorpdfstring{\protect\(F_{24}\protect\): Attractive sector/Rastrigin}{}}
\label{index:attractive-sector-rastrigin}\label{index:f24}
Combination of the attractive sector function (\href{https://coco.gforge.inria.fr/downloads/download16.00/bbobdocfunctions.pdf\#page=30}{\(f_6\) in the \bbob suite}) and the Rastrigin function
(\href{https://coco.gforge.inria.fr/downloads/download16.00/bbobdocfunctions.pdf\#page=75}{\(f_{15}\) in the \bbob suite}).

Both objectives are non-separable, and the second one
is highly multimodal (roughly \(10^n\) local
optima) while the first one is unimodal. Further
properties are that the first objective is highly
asymmetric and the second has a conditioning of about 10.

Contained in the \emph{moderate - multimodal} function group.

% #1: function string (2 digits)
% #2: dimension string (2 digits)
% #3: instance string (2 digits)
% #4: function number
% #5: dimension
\bbobfigs{24}{05}{01}{24}{5}


\subsection[\texorpdfstring{\protect\(F_{25}\protect\): Attractive sector/Schaffer F7, condition 10}{F25: Attractive sector/Schaffer F7, condition 10}]{\texorpdfstring{\protect\(F_{25}\protect\): Attractive sector/Schaffer F7, condition 10}{}}
\label{index:attractive-sector-schaffer-f7-condition-10}\label{index:f25}
Combination of the attractive sector function (\href{https://coco.gforge.inria.fr/downloads/download16.00/bbobdocfunctions.pdf\#page=30}{\(f_6\) in the \bbob suite}) and the Schaffer F7 function with condition number 10
(\href{https://coco.gforge.inria.fr/downloads/download16.00/bbobdocfunctions.pdf\#page=85}{\(f_{17}\) in the \bbob suite}).

Both objectives are non-separable and asymmetric.
While the first objective is unimodal, the second one is
a highly multimodal function with a low conditioning where
frequency and amplitude of the modulation vary.

Contained in the \emph{moderate - multimodal} function group.

% #1: function string (2 digits)
% #2: dimension string (2 digits)
% #3: instance string (2 digits)
% #4: function number
% #5: dimension
\bbobfigs{25}{05}{01}{25}{5}


\subsection[\texorpdfstring{\protect\(F_{26}\protect\): Attractive sector/Schwefel x*sin(x)}{F26: Attractive sector/Schwefel x*sin(x)}]{\texorpdfstring{\protect\(F_{26}\protect\): Attractive sector/Schwefel x*sin(x)}{}}
\label{index:f26}\label{index:attractive-sector-schwefel-x-sin-x}
Combination of the attractive sector function (\href{https://coco.gforge.inria.fr/downloads/download16.00/bbobdocfunctions.pdf\#page=30}{\(f_6\) in the \bbob suite}) and the Schwefel function (\href{https://coco.gforge.inria.fr/downloads/download16.00/bbobdocfunctions.pdf\#page=100}{\(f_{20}\) in the \bbob suite}).

The first objective is non-separable, unimodal, and asymmetric.
The second objective is partially separable and highly multimodal---having
the most prominent \(2^n\) minima located comparatively close to the
corners of the unpenalized search area.

Contained in the \emph{moderate - weakly-structured} function group.

% #1: function string (2 digits)
% #2: dimension string (2 digits)
% #3: instance string (2 digits)
% #4: function number
% #5: dimension
\bbobfigs{26}{05}{01}{26}{5}


\subsection[\texorpdfstring{\protect\(F_{27}\protect\): Attractive sector/Gallagher 101 peaks}{F27: Attractive sector/Gallagher 101 peaks}]{\texorpdfstring{\protect\(F_{27}\protect\): Attractive sector/Gallagher 101 peaks}{}}
\label{index:f27}\label{index:attractive-sector-gallagher-101-peaks}
Combination of the attractive sector function (\href{https://coco.gforge.inria.fr/downloads/download16.00/bbobdocfunctions.pdf\#page=30}{\(f_6\) in the \bbob suite}) and the Gallagher function with 101 peaks (\href{https://coco.gforge.inria.fr/downloads/download16.00/bbobdocfunctions.pdf\#page=105}{\(f_{21}\) in the \bbob suite}).

Both objective functions are non-separable but only the first
is unimodal. The first objective function is furthermore asymmetric.
The second objective function has 101 optima with position and height
being unrelated and randomly chosen (different for each instantiation
of the function). The conditioning around the global optimum of the second
objective function is about 30.

Contained in the \emph{moderate - weakly-structured} function group.

% #1: function string (2 digits)
% #2: dimension string (2 digits)
% #3: instance string (2 digits)
% #4: function number
% #5: dimension
\bbobfigs{27}{05}{01}{27}{5}


\subsection[\texorpdfstring{\protect\(F_{28}\protect\): Rosenbrock original/Rosenbrock original}{F28: Rosenbrock original/Rosenbrock original}]{\texorpdfstring{\protect\(F_{28}\protect\): Rosenbrock original/Rosenbrock original}{}}
\label{index:rosenbrock-original-rosenbrock-original}\label{index:f28}
Combination of two Rosenbrock functions (\href{https://coco.gforge.inria.fr/downloads/download16.00/bbobdocfunctions.pdf\#page=40}{\(f_8\) in the \bbob suite}).

Both objectives are partially separable (tri-band structure) and have
a local optimum with an attraction volume of about 25\%.

Contained in the \emph{moderate - moderate} function group.

% #1: function string (2 digits)
% #2: dimension string (2 digits)
% #3: instance string (2 digits)
% #4: function number
% #5: dimension
\bbobfigs{28}{05}{01}{28}{5}


\subsection[\texorpdfstring{\protect\(F_{29}\protect\): Rosenbrock original/Sharp ridge}{F29: Rosenbrock original/Sharp ridge}]{\texorpdfstring{\protect\(F_{29}\protect\): Rosenbrock original/Sharp ridge}{}}
\label{index:rosenbrock-original-sharp-ridge}\label{index:f29}
Combination of the Rosenbrock function (\href{https://coco.gforge.inria.fr/downloads/download16.00/bbobdocfunctions.pdf\#page=40}{\(f_8\) in the \bbob suite}) and the
sharp ridge function (\href{https://coco.gforge.inria.fr/downloads/download16.00/bbobdocfunctions.pdf\#page=65}{\(f_{13}\) in the \bbob suite}).

The first objective function is partially separable (tri-band structure)
and has a local optimum with an attraction volume of about 25\%.
The second objective is unimodal and non-separable and, for
optimizing it, a sharp, i.e., non-differentiable ridge has to be followed.

Contained in the \emph{moderate - ill-conditioned} function group.

% #1: function string (2 digits)
% #2: dimension string (2 digits)
% #3: instance string (2 digits)
% #4: function number
% #5: dimension
\bbobfigs{29}{05}{01}{29}{5}


\subsection[\texorpdfstring{\protect\(F_{30}\protect\): Rosenbrock original/Sum of Different Powers}{F30: Rosenbrock original/Sum of Different Powers}]{\texorpdfstring{\protect\(F_{30}\protect\): Rosenbrock original/Sum of Different Powers}{}}
\label{index:f30}\label{index:rosenbrock-original-sum-of-different-powers}
Combination of the Rosenbrock function (\href{https://coco.gforge.inria.fr/downloads/download16.00/bbobdocfunctions.pdf\#page=40}{\(f_8\) in the \bbob suite}) and the sum of different powers function
(\href{https://coco.gforge.inria.fr/downloads/download16.00/bbobdocfunctions.pdf\#page=70}{\(f_{14}\) in the \bbob suite}).

The first objective function is partially separable (tri-band structure)
and has a local optimum with an attraction volume of about 25\%.
The second objective function is unimodal and non-separable. When
approaching the second objective's optimum, the sensitivities of the
variables in the rotated search space become more and more different.

Contained in the \emph{moderate - ill-conditioned} function group.

% #1: function string (2 digits)
% #2: dimension string (2 digits)
% #3: instance string (2 digits)
% #4: function number
% #5: dimension
\bbobfigs{30}{05}{01}{30}{5}


\subsection[\texorpdfstring{\protect\(F_{31}\protect\): Rosenbrock original/Rastrigin}{F31: Rosenbrock original/Rastrigin}]{\texorpdfstring{\protect\(F_{31}\protect\): Rosenbrock original/Rastrigin}{}}
\label{index:f31}\label{index:rosenbrock-original-rastrigin}
Combination of the Rosenbrock function (\href{https://coco.gforge.inria.fr/downloads/download16.00/bbobdocfunctions.pdf\#page=40}{\(f_8\) in the \bbob suite}) and the Rastrigin function
(\href{https://coco.gforge.inria.fr/downloads/download16.00/bbobdocfunctions.pdf\#page=75}{\(f_{15}\) in the \bbob suite}).

The first objective function is partially separable (tri-band structure)
and has a local optimum with an attraction volume of about 25\%.
The second objective function is non-separable and
highly multimodal (roughly \(10^n\) local
optima).

Contained in the \emph{moderate - multimodal} function group.

% #1: function string (2 digits)
% #2: dimension string (2 digits)
% #3: instance string (2 digits)
% #4: function number
% #5: dimension
\bbobfigs{31}{05}{01}{31}{5}


\subsection[\texorpdfstring{\protect\(F_{32}\protect\): Rosenbrock original/Schaffer F7, condition 10}{F32: Rosenbrock original/Schaffer F7, condition 10}]{\texorpdfstring{\protect\(F_{32}\protect\): Rosenbrock original/Schaffer F7, condition 10}{}}
\label{index:f32}\label{index:rosenbrock-original-schaffer-f7-condition-10}
Combination of the Rosenbrock function (\href{https://coco.gforge.inria.fr/downloads/download16.00/bbobdocfunctions.pdf\#page=40}{\(f_8\) in the \bbob suite}) and the
Schaffer F7 function with condition number 10
(\href{https://coco.gforge.inria.fr/downloads/download16.00/bbobdocfunctions.pdf\#page=85}{\(f_{17}\) in the \bbob suite}).

The first objective function is partially separable (tri-band structure)
and has a local optimum with an attraction volume of about 25\%.
The second objective function is non-separable, asymmetric, and
highly multimodal with a low conditioning where
frequency and amplitude of the modulation vary.

Contained in the \emph{moderate - multimodal} function group.

% #1: function string (2 digits)
% #2: dimension string (2 digits)
% #3: instance string (2 digits)
% #4: function number
% #5: dimension
\bbobfigs{32}{05}{01}{32}{5}


\subsection[\texorpdfstring{\protect\(F_{33}\protect\): Rosenbrock original/Schwefel x*sin(x)}{F33: Rosenbrock original/Schwefel x*sin(x)}]{\texorpdfstring{\protect\(F_{33}\protect\): Rosenbrock original/Schwefel x*sin(x)}{}}
\label{index:rosenbrock-original-schwefel-x-sin-x}\label{index:f33}
Combination of the Rosenbrock function (\href{https://coco.gforge.inria.fr/downloads/download16.00/bbobdocfunctions.pdf\#page=40}{\(f_8\) in the \bbob suite}) and the
Schwefel function (\href{https://coco.gforge.inria.fr/downloads/download16.00/bbobdocfunctions.pdf\#page=100}{\(f_{20}\) in the \bbob suite}).

Both objective functions are partially separable.
While the first objective function has a local optimum with an attraction
volume of about 25\%, the second objective function is highly
multimodal---having the most prominent \(2^n\) minima located
comparatively close to the corners of its unpenalized search area.

Contained in the \emph{moderate - weakly-structured} function group.

% #1: function string (2 digits)
% #2: dimension string (2 digits)
% #3: instance string (2 digits)
% #4: function number
% #5: dimension
\bbobfigs{33}{05}{01}{33}{5}


\subsection[\texorpdfstring{\protect\(F_{34}\protect\): Rosenbrock original/Gallagher 101 peaks}{F34: Rosenbrock original/Gallagher 101 peaks}]{\texorpdfstring{\protect\(F_{34}\protect\): Rosenbrock original/Gallagher 101 peaks}{}}
\label{index:f34}\label{index:rosenbrock-original-gallagher-101-peaks}
Combination of the Rosenbrock function (\href{https://coco.gforge.inria.fr/downloads/download16.00/bbobdocfunctions.pdf\#page=40}{\(f_8\) in the \bbob suite}) and
the Gallagher function with 101 peaks (\href{https://coco.gforge.inria.fr/downloads/download16.00/bbobdocfunctions.pdf\#page=105}{\(f_{21}\) in the \bbob suite}).

The first objective function is partially separable, the second one
non-separable. While the first objective function has a local optimum
with an attraction volume of about 25\%, the second objective function
has 101 optima with position and height being unrelated and randomly
chosen (different for each instantiation of the function). The
conditioning around the global optimum of the second objective function
is about 30.

Contained in the \emph{moderate - weakly-structured} function group.

% #1: function string (2 digits)
% #2: dimension string (2 digits)
% #3: instance string (2 digits)
% #4: function number
% #5: dimension
\bbobfigs{34}{05}{01}{34}{5}


\subsection[\texorpdfstring{\protect\(F_{35}\protect\): Sharp ridge/Sharp ridge}{F35: Sharp ridge/Sharp ridge}]{\texorpdfstring{\protect\(F_{35}\protect\): Sharp ridge/Sharp ridge}{}}
\label{index:sharp-ridge-sharp-ridge}\label{index:f35}
Combination of two sharp ridge functions (\href{https://coco.gforge.inria.fr/downloads/download16.00/bbobdocfunctions.pdf\#page=65}{\(f_{13}\) in the \bbob suite}).

Both objective functions are unimodal and non-separable and, for
optimizing them, two sharp, i.e., non-differentiable ridges have to be
followed.

Contained in the \emph{ill-conditioned - ill-conditioned} function group.

% #1: function string (2 digits)
% #2: dimension string (2 digits)
% #3: instance string (2 digits)
% #4: function number
% #5: dimension
\bbobfigs{35}{05}{01}{35}{5}


\subsection[\texorpdfstring{\protect\(F_{36}\protect\): Sharp ridge/Sum of Different Powers}{F36: Sharp ridge/Sum of Different Powers}]{\texorpdfstring{\protect\(F_{36}\protect\): Sharp ridge/Sum of Different Powers}{}}
\label{index:sharp-ridge-sum-of-different-powers}\label{index:f36}
Combination of the sharp ridge function (\href{https://coco.gforge.inria.fr/downloads/download16.00/bbobdocfunctions.pdf\#page=65}{\(f_{13}\) in the \bbob suite}) and the
sum of different powers function
(\href{https://coco.gforge.inria.fr/downloads/download16.00/bbobdocfunctions.pdf\#page=70}{\(f_{14}\) in the \bbob suite}).

Both functions are uni-modal and non-separable.
For optimizing the first objective, a sharp, i.e., non-differentiable
ridge has to be followed.
When approaching the second objective's optimum, the sensitivities of the
variables in the rotated search space become more and more different.

Contained in the \emph{ill-conditioned - ill-conditioned} function group.

% #1: function string (2 digits)
% #2: dimension string (2 digits)
% #3: instance string (2 digits)
% #4: function number
% #5: dimension
\bbobfigs{36}{05}{01}{36}{5}


\subsection[\texorpdfstring{\protect\(F_{37}\protect\): Sharp ridge/Rastrigin}{F37: Sharp ridge/Rastrigin}]{\texorpdfstring{\protect\(F_{37}\protect\): Sharp ridge/Rastrigin}{}}
\label{index:sharp-ridge-rastrigin}\label{index:f37}
Combination of the sharp ridge function (\href{https://coco.gforge.inria.fr/downloads/download16.00/bbobdocfunctions.pdf\#page=65}{\(f_{13}\) in the \bbob suite}) and the Rastrigin function
(\href{https://coco.gforge.inria.fr/downloads/download16.00/bbobdocfunctions.pdf\#page=75}{\(f_{15}\) in the \bbob suite}).

Both functions are non-separable. While the first one
is unimodal and non-differentiable at its ridge, the second objective
function is highly multimodal (roughly \(10^n\) local optima).

Contained in the \emph{ill-conditioned - multimodal} function group.

% #1: function string (2 digits)
% #2: dimension string (2 digits)
% #3: instance string (2 digits)
% #4: function number
% #5: dimension
\bbobfigs{37}{05}{01}{37}{5}


\subsection[\texorpdfstring{\protect\(F_{38}\protect\): Sharp ridge/Schaffer F7, condition 10}{F38: Sharp ridge/Schaffer F7, condition 10}]{\texorpdfstring{\protect\(F_{38}\protect\): Sharp ridge/Schaffer F7, condition 10}{}}
\label{index:f38}\label{index:sharp-ridge-schaffer-f7-condition-10}
Combination of the sharp ridge function (\href{https://coco.gforge.inria.fr/downloads/download16.00/bbobdocfunctions.pdf\#page=65}{\(f_{13}\) in the \bbob suite}) and the
Schaffer F7 function with condition number 10
(\href{https://coco.gforge.inria.fr/downloads/download16.00/bbobdocfunctions.pdf\#page=85}{\(f_{17}\) in the \bbob suite}).

Both functions are non-separable. While the first one
is unimodal and non-differentiable at its ridge, the second objective
function is asymmetric and highly multimodal with a low conditioning where
frequency and amplitude of the modulation vary.

Contained in the \emph{ill-conditioned - multimodal} function group.

% #1: function string (2 digits)
% #2: dimension string (2 digits)
% #3: instance string (2 digits)
% #4: function number
% #5: dimension
\bbobfigs{38}{05}{01}{38}{5}


\subsection[\texorpdfstring{\protect\(F_{39}\protect\): Sharp ridge/Schwefel x*sin(x)}{F39: Sharp ridge/Schwefel x*sin(x)}]{\texorpdfstring{\protect\(F_{39}\protect\): Sharp ridge/Schwefel x*sin(x)}{}}
\label{index:f39}\label{index:sharp-ridge-schwefel-x-sin-x}
Combination of the sharp ridge function (\href{https://coco.gforge.inria.fr/downloads/download16.00/bbobdocfunctions.pdf\#page=65}{\(f_{13}\) in the \bbob suite}) and the
Schwefel function (\href{https://coco.gforge.inria.fr/downloads/download16.00/bbobdocfunctions.pdf\#page=100}{\(f_{20}\) in the \bbob suite}).

While the first objective function is unimodal, non-separable, and
non-differentiable at its ridge, the second objective function is highly
multimodal---having the most prominent \(2^n\) minima located
comparatively close to the corners of its unpenalized search area.

Contained in the \emph{ill-conditioned - weakly-structured} function group.

% #1: function string (2 digits)
% #2: dimension string (2 digits)
% #3: instance string (2 digits)
% #4: function number
% #5: dimension
\bbobfigs{39}{05}{01}{39}{5}


\subsection[\texorpdfstring{\protect\(F_{40}\protect\): Sharp ridge/Gallagher 101 peaks}{F40: Sharp ridge/Gallagher 101 peaks}]{\texorpdfstring{\protect\(F_{40}\protect\): Sharp ridge/Gallagher 101 peaks}{}}
\label{index:f40}\label{index:sharp-ridge-gallagher-101-peaks}
Combination of the sharp ridge function (\href{https://coco.gforge.inria.fr/downloads/download16.00/bbobdocfunctions.pdf\#page=65}{\(f_{13}\) in the \bbob suite}) and the
Gallagher function with 101 peaks (\href{https://coco.gforge.inria.fr/downloads/download16.00/bbobdocfunctions.pdf\#page=105}{\(f_{21}\) in the \bbob suite}).

Both objective functions are non-separable.
While the first objective function is unimodal and non-differentiable at
its ridge, the second objective function
has 101 optima with position and height being unrelated and randomly
chosen (different for each instantiation of the function). The
conditioning around the global optimum of the second objective function
is about 30.

Contained in the \emph{ill-conditioned - weakly-structured} function group.

% #1: function string (2 digits)
% #2: dimension string (2 digits)
% #3: instance string (2 digits)
% #4: function number
% #5: dimension
\bbobfigs{40}{05}{01}{40}{5}


\subsection[\texorpdfstring{\protect\(F_{41}\protect\): Sum of Different Powers/Sum of Different Powers}{F41: Sum of Different Powers/Sum of Different Powers}]{\texorpdfstring{\protect\(F_{41}\protect\): Sum of Different Powers/Sum of Different Powers}{}}
\label{index:f41}\label{index:sum-of-different-powers-sum-of-different-powers}
Combination of two sum of different powers functions
(\href{https://coco.gforge.inria.fr/downloads/download16.00/bbobdocfunctions.pdf\#page=70}{\(f_{14}\) in the \bbob suite}).

Both functions are uni-modal and non-separable where the sensitivities of
the variables in the rotated search space become more and more different
when approaching the objectives' optima.

Contained in the \emph{ill-conditioned - ill-conditioned} function group.

% #1: function string (2 digits)
% #2: dimension string (2 digits)
% #3: instance string (2 digits)
% #4: function number
% #5: dimension
\bbobfigs{41}{05}{01}{41}{5}


\subsection[\texorpdfstring{\protect\(F_{42}\protect\): Sum of Different Powers/Rastrigin}{F42: Sum of Different Powers/Rastrigin}]{\texorpdfstring{\protect\(F_{42}\protect\): Sum of Different Powers/Rastrigin}{}}
\label{index:f42}\label{index:sum-of-different-powers-rastrigin}
Combination of the sum of different powers functions
(\href{https://coco.gforge.inria.fr/downloads/download16.00/bbobdocfunctions.pdf\#page=70}{\(f_{14}\) in the \bbob suite}) and the Rastrigin function
(\href{https://coco.gforge.inria.fr/downloads/download16.00/bbobdocfunctions.pdf\#page=75}{\(f_{15}\) in the \bbob suite}).

Both objective functions are non-separable. While the first one
is unimodal, the second objective
function is highly multimodal (roughly \(10^n\) local optima).

Contained in the \emph{ill-conditioned - multimodal} function group.

% #1: function string (2 digits)
% #2: dimension string (2 digits)
% #3: instance string (2 digits)
% #4: function number
% #5: dimension
\bbobfigs{42}{05}{01}{42}{5}


\subsection[\texorpdfstring{\protect\(F_{43}\protect\): Sum of Different Powers/Schaffer F7, condition 10}{F43: Sum of Different Powers/Schaffer F7, condition 10}]{\texorpdfstring{\protect\(F_{43}\protect\): Sum of Different Powers/Schaffer F7, condition 10}{}}
\label{index:sum-of-different-powers-schaffer-f7-condition-10}\label{index:f43}
Combination of the sum of different powers functions
(\href{https://coco.gforge.inria.fr/downloads/download16.00/bbobdocfunctions.pdf\#page=70}{\(f_{14}\) in the \bbob suite}) and the Schaffer F7 function with
condition number 10 (\href{https://coco.gforge.inria.fr/downloads/download16.00/bbobdocfunctions.pdf\#page=85}{\(f_{17}\) in the \bbob suite}).

Both objective functions are non-separable. While the first one
is unimodal with an increasing conditioning once the optimum is approached,
the second objective function is asymmetric and highly multimodal with a
low conditioning where frequency and amplitude of the modulation vary.

Contained in the \emph{ill-conditioned - multimodal} function group.

% #1: function string (2 digits)
% #2: dimension string (2 digits)
% #3: instance string (2 digits)
% #4: function number
% #5: dimension
\bbobfigs{43}{05}{01}{43}{5}


\subsection[\texorpdfstring{\protect\(F_{44}\protect\): Sum of Different Powers/Schwefel x*sin(x)}{F44: Sum of Different Powers/Schwefel x*sin(x)}]{\texorpdfstring{\protect\(F_{44}\protect\): Sum of Different Powers/Schwefel x*sin(x)}{}}
\label{index:f44}\label{index:sum-of-different-powers-schwefel-x-sin-x}
Combination of the sum of different powers functions
(\href{https://coco.gforge.inria.fr/downloads/download16.00/bbobdocfunctions.pdf\#page=70}{\(f_{14}\) in the \bbob suite}) and the Schwefel function (\href{https://coco.gforge.inria.fr/downloads/download16.00/bbobdocfunctions.pdf\#page=100}{\(f_{20}\) in the \bbob suite}).

Both objectives are non-separable.
While the first objective function is unimodal,
the second objective function is highly multimodal---having the most
prominent \(2^n\) minima located comparatively close to the corners
of its unpenalized search area.

Contained in the \emph{ill-conditioned - weakly-structured} function group.

% #1: function string (2 digits)
% #2: dimension string (2 digits)
% #3: instance string (2 digits)
% #4: function number
% #5: dimension
\bbobfigs{44}{05}{01}{44}{5}


\subsection[\texorpdfstring{\protect\(F_{45}\protect\): Sum of Different Powers/Gallagher 101 peaks}{F45: Sum of Different Powers/Gallagher 101 peaks}]{\texorpdfstring{\protect\(F_{45}\protect\): Sum of Different Powers/Gallagher 101 peaks}{}}
\label{index:sum-of-different-powers-gallagher-101-peaks}\label{index:f45}
Combination of the sum of different powers functions
(\href{https://coco.gforge.inria.fr/downloads/download16.00/bbobdocfunctions.pdf\#page=70}{\(f_{14}\) in the \bbob suite}) and the Gallagher function with
101 peaks (\href{https://coco.gforge.inria.fr/downloads/download16.00/bbobdocfunctions.pdf\#page=105}{\(f_{21}\) in the \bbob suite}).

Both objective functions are non-separable.
While the first objective function is unimodal, the second objective function
has 101 optima with position and height being unrelated and randomly
chosen (different for each instantiation of the function). The
conditioning around the global optimum of the second objective function
is about 30.

Contained in the \emph{ill-conditioned - weakly-structured} function group.

% #1: function string (2 digits)
% #2: dimension string (2 digits)
% #3: instance string (2 digits)
% #4: function number
% #5: dimension
\bbobfigs{45}{05}{01}{45}{5}


\subsection[\texorpdfstring{\protect\(F_{46}\protect\): Rastrigin/Rastrigin}{F46: Rastrigin/Rastrigin}]{\texorpdfstring{\protect\(F_{46}\protect\): Rastrigin/Rastrigin}{}}
\label{index:rastrigin-rastrigin}\label{index:f46}
Combination of two Rastrigin functions
(\href{https://coco.gforge.inria.fr/downloads/download16.00/bbobdocfunctions.pdf\#page=75}{\(f_{15}\) in the \bbob suite}).

Both objective functions are non-separable and highly multimodal
(roughly \(10^n\) local optima).

Contained in the \emph{multimodal - multimodal} function group.

% #1: function string (2 digits)
% #2: dimension string (2 digits)
% #3: instance string (2 digits)
% #4: function number
% #5: dimension
\bbobfigs{46}{05}{01}{46}{5}


\subsection[\texorpdfstring{\protect\(F_{47}\protect\): Rastrigin/Schaffer F7, condition 10}{F47: Rastrigin/Schaffer F7, condition 10}]{\texorpdfstring{\protect\(F_{47}\protect\): Rastrigin/Schaffer F7, condition 10}{}}
\label{index:rastrigin-schaffer-f7-condition-10}\label{index:f47}
Combination of the Rastrigin function
(\href{https://coco.gforge.inria.fr/downloads/download16.00/bbobdocfunctions.pdf\#page=75}{\(f_{15}\) in the \bbob suite}) and the Schaffer F7 function with
condition number 10 (\href{https://coco.gforge.inria.fr/downloads/download16.00/bbobdocfunctions.pdf\#page=85}{\(f_{17}\) in the \bbob suite}).

Both objective functions are non-separable and highly multimodal.

Contained in the \emph{multimodal - multimodal} function group.

% #1: function string (2 digits)
% #2: dimension string (2 digits)
% #3: instance string (2 digits)
% #4: function number
% #5: dimension
\bbobfigs{47}{05}{01}{47}{5}


\subsection[\texorpdfstring{\protect\(F_{48}\protect\): Rastrigin/Schwefel x*sin(x)}{F48: Rastrigin/Schwefel x*sin(x)}]{\texorpdfstring{\protect\(F_{48}\protect\): Rastrigin/Schwefel x*sin(x)}{}}
\label{index:f48}\label{index:rastrigin-schwefel-x-sin-x}
Combination of the Rastrigin function
(\href{https://coco.gforge.inria.fr/downloads/download16.00/bbobdocfunctions.pdf\#page=75}{\(f_{15}\) in the \bbob suite}) and the Schwefel function (\href{https://coco.gforge.inria.fr/downloads/download16.00/bbobdocfunctions.pdf\#page=100}{\(f_{20}\) in the \bbob suite}).

Both objective functions are non-separable and highly multimodal where
the first has roughly \(10^n\) local optima and the most prominent
\(2^n\) minima of the second objective function are located
comparatively close to the corners of its unpenalized search area.

Contained in the \emph{multimodal - weakly-structured} function group.

% #1: function string (2 digits)
% #2: dimension string (2 digits)
% #3: instance string (2 digits)
% #4: function number
% #5: dimension
\bbobfigs{48}{05}{01}{48}{5}


\subsection[\texorpdfstring{\protect\(F_{49}\protect\): Rastrigin/Gallagher 101 peaks}{F49: Rastrigin/Gallagher 101 peaks}]{\texorpdfstring{\protect\(F_{49}\protect\): Rastrigin/Gallagher 101 peaks}{}}
\label{index:f49}\label{index:rastrigin-gallagher-101-peaks}
Combination of the Rastrigin function
(\href{https://coco.gforge.inria.fr/downloads/download16.00/bbobdocfunctions.pdf\#page=75}{\(f_{15}\) in the \bbob suite}) and the Gallagher function with
101 peaks (\href{https://coco.gforge.inria.fr/downloads/download16.00/bbobdocfunctions.pdf\#page=105}{\(f_{21}\) in the \bbob suite}).

Both objective functions are non-separable and highly multimodal where
the first has roughly \(10^n\) local optima and the second has
101 optima with position and height being unrelated and randomly
chosen (different for each instantiation of the function).

Contained in the \emph{multimodal - weakly-structured} function group.

% #1: function string (2 digits)
% #2: dimension string (2 digits)
% #3: instance string (2 digits)
% #4: function number
% #5: dimension
\bbobfigs{49}{05}{01}{49}{5}


\subsection[\texorpdfstring{\protect\(F_{50}\protect\): Schaffer F7, condition 10/Schaffer F7, condition 10}{F50: Schaffer F7, condition 10/Schaffer F7, condition 10}]{\texorpdfstring{\protect\(F_{50}\protect\): Schaffer F7, condition 10/Schaffer F7, condition 10}{}}
\label{index:schaffer-f7-condition-10-schaffer-f7-condition-10}\label{index:f50}
Combination of two Schaffer F7 functions with
condition number 10 (\href{https://coco.gforge.inria.fr/downloads/download16.00/bbobdocfunctions.pdf\#page=85}{\(f_{17}\) in the \bbob suite}).

Both objective functions are non-separable and highly multimodal.

Contained in the \emph{multimodal - multimodal} function group.

% #1: function string (2 digits)
% #2: dimension string (2 digits)
% #3: instance string (2 digits)
% #4: function number
% #5: dimension
\bbobfigs{50}{05}{01}{50}{5}


\subsection[\texorpdfstring{\protect\(F_{51}\protect\): Schaffer F7, condition 10/Schwefel x*sin(x)}{F51: Schaffer F7, condition 10/Schwefel x*sin(x)}]{\texorpdfstring{\protect\(F_{51}\protect\): Schaffer F7, condition 10/Schwefel x*sin(x)}{}}
\label{index:schaffer-f7-condition-10-schwefel-x-sin-x}\label{index:f51}
Combination of the Schaffer F7 function with
condition number 10 (\href{https://coco.gforge.inria.fr/downloads/download16.00/bbobdocfunctions.pdf\#page=85}{\(f_{17}\) in the \bbob suite})
and the Schwefel function (\href{https://coco.gforge.inria.fr/downloads/download16.00/bbobdocfunctions.pdf\#page=100}{\(f_{20}\) in the \bbob suite}).

Both objective functions are non-separable and highly multimodal.
While frequency and amplitude of the modulation vary in an almost
regular fashion in the first objective function, the second objective
function possesses less global structure.

Contained in the \emph{multimodal - weakly-structured} function group.

% #1: function string (2 digits)
% #2: dimension string (2 digits)
% #3: instance string (2 digits)
% #4: function number
% #5: dimension
\bbobfigs{51}{05}{01}{51}{5}


\subsection[\texorpdfstring{\protect\(F_{52}\protect\): Schaffer F7, condition 10/Gallagher 101 peaks}{F52: Schaffer F7, condition 10/Gallagher 101 peaks}]{\texorpdfstring{\protect\(F_{52}\protect\): Schaffer F7, condition 10/Gallagher 101 peaks}{}}
\label{index:schaffer-f7-condition-10-gallagher-101-peaks}\label{index:f52}
Combination of the Schaffer F7 function with
condition number 10 (\href{https://coco.gforge.inria.fr/downloads/download16.00/bbobdocfunctions.pdf\#page=85}{\(f_{17}\) in the \bbob suite})
and the Gallagher function with
101 peaks (\href{https://coco.gforge.inria.fr/downloads/download16.00/bbobdocfunctions.pdf\#page=105}{\(f_{21}\) in the \bbob suite}).

Both objective functions are non-separable and highly multimodal.
While frequency and amplitude of the modulation vary in an almost
regular fashion in the first objective function, the second has
101 optima with position and height being unrelated and randomly
chosen (different for each instantiation of the function).

Contained in the \emph{multimodal - weakly-structured} function group.

% #1: function string (2 digits)
% #2: dimension string (2 digits)
% #3: instance string (2 digits)
% #4: function number
% #5: dimension
\bbobfigs{52}{05}{01}{52}{5}


\subsection[\texorpdfstring{\protect\(F_{53}\protect\): Schwefel x*sin(x)/Schwefel x*sin(x)}{F53: Schwefel x*sin(x)/Schwefel x*sin(x)}]{\texorpdfstring{\protect\(F_{53}\protect\): Schwefel x*sin(x)/Schwefel x*sin(x)}{}}
\label{index:schwefel-x-sin-x-schwefel-x-sin-x}\label{index:f53}
Combination of two Schwefel functions (\href{https://coco.gforge.inria.fr/downloads/download16.00/bbobdocfunctions.pdf\#page=100}{\(f_{20}\) in the \bbob suite}).

Both objective functions are non-separable and highly multimodal where
the most prominent \(2^n\) minima of each objective function are
located comparatively close to the corners of its unpenalized search area.
Due to the combinatorial nature of the Schwefel function, it is likely
in low dimensions that the Pareto set goes through the origin of the
search space.

Contained in the \emph{weakly-structured - weakly-structured} function group.

% #1: function string (2 digits)
% #2: dimension string (2 digits)
% #3: instance string (2 digits)
% #4: function number
% #5: dimension
\bbobfigs{53}{05}{01}{53}{5}


\subsection[\texorpdfstring{\protect\(F_{54}\protect\): Schwefel x*sin(x)/Gallagher 101 peaks}{F54: Schwefel x*sin(x)/Gallagher 101 peaks}]{\texorpdfstring{\protect\(F_{54}\protect\): Schwefel x*sin(x)/Gallagher 101 peaks}{}}
\label{index:f54}\label{index:schwefel-x-sin-x-gallagher-101-peaks}
Combination of the Schwefel function (\href{https://coco.gforge.inria.fr/downloads/download16.00/bbobdocfunctions.pdf\#page=100}{\(f_{20}\) in the \bbob suite}) and the Gallagher function with
101 peaks (\href{https://coco.gforge.inria.fr/downloads/download16.00/bbobdocfunctions.pdf\#page=105}{\(f_{21}\) in the \bbob suite}).

Both objective functions are non-separable and highly multimodal.
For the first objective function, the most prominent \(2^n\) minima
are located comparatively close to the corners of its unpenalized search
area. For the second objective, position and height of all
101 optima are unrelated and randomly
chosen (different for each instantiation of the function).

Contained in the \emph{weakly-structured - weakly-structured} function group.

% #1: function string (2 digits)
% #2: dimension string (2 digits)
% #3: instance string (2 digits)
% #4: function number
% #5: dimension
\bbobfigs{54}{05}{01}{54}{5}


\subsection[\texorpdfstring{\protect\(F_{55}\protect\): Gallagher 101 peaks/Gallagher 101 peaks}{F55: Gallagher 101 peaks/Gallagher 101 peaks}]{\texorpdfstring{\protect\(F_{55}\protect\): Gallagher 101 peaks/Gallagher 101 peaks}{}}
\label{index:gallagher-101-peaks-gallagher-101-peaks}\label{index:f55}
Combination of two Gallagher functions with
101 peaks (\href{https://coco.gforge.inria.fr/downloads/download16.00/bbobdocfunctions.pdf\#page=105}{\(f_{21}\) in the \bbob suite}).

Both objective functions are non-separable and highly multimodal.
Position and height of all 101 optima in each objective function
are unrelated and randomly chosen and thus, no global structure
is present.

Contained in the \emph{weakly-structured - weakly-structured} function group.

% #1: function string (2 digits)
% #2: dimension string (2 digits)
% #3: instance string (2 digits)
% #4: function number
% #5: dimension
\bbobfigs{55}{05}{01}{55}{5}


\subsection[\texorpdfstring{\protect\(F_{56}\protect\): Sphere/Rastrigin separable}{F56: Sphere/Rastrigin separable}]{\texorpdfstring{\protect\(F_{56}\protect\): Sphere/Rastrigin separable}{}}
\label{index:f56}\label{index:sphere-rastrigin-separable}
Combination of the Sphere function (\href{https://coco.gforge.inria.fr/downloads/download16.00/bbobdocfunctions.pdf\#page=5}{\(f_1\) in the \bbob suite}) and the
separable Rastrigin function (\href{https://coco.gforge.inria.fr/downloads/download16.00/bbobdocfunctions.pdf\#page=15}{\(f_3\) in the \bbob suite}).

While the first objective function is unimodal, highly symmetric,
rotational and scale invariant, the second one is highly multimodal
with a comparatively regular structure for the placement of the optima.
Note that the non-linear transformations of the second objective's
Rastrigin function alleviate the symmetry and regularity of the
original Rastrigin function.

Contained in the \emph{separable - separable} function group.

% #1: function string (2 digits)
% #2: dimension string (2 digits)
% #3: instance string (2 digits)
% #4: function number
% #5: dimension
\bbobfigs{56}{05}{01}{56}{5}


\subsection[\texorpdfstring{\protect\(F_{57}\protect\): Sphere/Rastrigin-B\"{u}che}{F57: Sphere/Rastrigin-B\"{u}che}]{\texorpdfstring{\protect\(F_{57}\protect\): Sphere/Rastrigin-B\"{u}che}{}}
\label{index:f57}\label{index:sphere-rastrigin-buche}
Combination of the Sphere function (\href{https://coco.gforge.inria.fr/downloads/download16.00/bbobdocfunctions.pdf\#page=5}{\(f_1\) in the \bbob suite}) and the
separable B\"{u}che-Rastrigin function (\href{https://coco.gforge.inria.fr/downloads/download16.00/bbobdocfunctions.pdf\#page=20}{\(f_4\) in the \bbob suite}).

While the first objective function is unimodal, highly symmetric,
rotational and scale invariant, the second one is highly multimodal
with a structured but highly asymmetric placement of the optima.
Constructed as a deceptive function for symmetrically distributed search
operators, the second objective function has roughly 10D
local optima, a conditioning of about 10, and a skew factor of about
10 in x-space and 100 in f-space.

Contained in the \emph{separable - separable} function group.

% #1: function string (2 digits)
% #2: dimension string (2 digits)
% #3: instance string (2 digits)
% #4: function number
% #5: dimension
\bbobfigs{57}{05}{01}{57}{5}


\subsection[\texorpdfstring{\protect\(F_{58}\protect\): Sphere/Linear slope}{F58: Sphere/Linear slope}]{\texorpdfstring{\protect\(F_{58}\protect\): Sphere/Linear slope}{}}
\label{index:sphere-linear-slope}\label{index:f58}
Combination of the Sphere function (\href{https://coco.gforge.inria.fr/downloads/download16.00/bbobdocfunctions.pdf\#page=5}{\(f_1\) in the \bbob suite}) and the
Linear Slope function (\href{https://coco.gforge.inria.fr/downloads/download16.00/bbobdocfunctions.pdf\#page=25}{\(f_5\) in the \bbob suite}).

Both objective functions are separable and amongst the simplest
continuous functions to optimize. The first objective function is fully
quadratic and symmetric around the optimum, the second objective function
is fully linear within the hypercube \([-5,5]^n\) and has a region of
constant \(f\)-value outside the hypercube by definition to ensure
that a solution at one corner of \([-5,5]^n\) has optimal function value.

Contained in the \emph{separable - separable} function group.

% #1: function string (2 digits)
% #2: dimension string (2 digits)
% #3: instance string (2 digits)
% #4: function number
% #5: dimension
\bbobfigs{58}{05}{01}{58}{5}


\subsection[\texorpdfstring{\protect\(F_{59}\protect\): Separable Ellipsoid/Separable Rastrigin}{F59: Separable Ellipsoid/Separable Rastrigin}]{\texorpdfstring{\protect\(F_{59}\protect\): Separable Ellipsoid/Separable Rastrigin}{}}
\label{index:separable-ellipsoid-separable-rastrigin}\label{index:f59}
Combination of the separable Ellipsoid function (\href{https://coco.gforge.inria.fr/downloads/download16.00/bbobdocfunctions.pdf\#page=10}{\(f_2\) in the \bbob suite}) and the
separable Rastrigin function (\href{https://coco.gforge.inria.fr/downloads/download16.00/bbobdocfunctions.pdf\#page=15}{\(f_3\) in the \bbob suite}).

Besides being both separable, the two objective functions are quite opposite:
the first objective function is unimodal, globally quadratic and ill-conditioned
with a conditioning of about \(10^6\) with smooth local irregularities while
the second objective function is highly multimodal with roughly \(10n\) local
optima and only small conditioning of about \(10\). Note that the separable
Rastrigin function has a comparatively regular structure for the
placement of the optima but asymmetric and oscillating non-linear transformations
of this function alleviates the symmetry and regularity of the original Rastrigin function.

Contained in the \emph{separable - separable} function group.

% #1: function string (2 digits)
% #2: dimension string (2 digits)
% #3: instance string (2 digits)
% #4: function number
% #5: dimension
\bbobfigs{59}{05}{01}{59}{5}


\subsection[\texorpdfstring{\protect\(F_{60}\protect\): separable Ellipsoid/B\"{u}che-Rastrigin}{F60: separable Ellipsoid/B\"{u}che-Rastrigin}]{\texorpdfstring{\protect\(F_{60}\protect\): separable Ellipsoid/B\"{u}che-Rastrigin}{}}
\label{index:separable-ellipsoid-buche-rastrigin}\label{index:f60}
Combination of the separable Ellipsoid function (\href{https://coco.gforge.inria.fr/downloads/download16.00/bbobdocfunctions.pdf\#page=10}{\(f_2\) in the \bbob suite}) and the
separable B\"{u}che-Rastrigin function (\href{https://coco.gforge.inria.fr/downloads/download16.00/bbobdocfunctions.pdf\#page=20}{\(f_4\) in the \bbob suite}).

Besides being both separable, the two objective functions are quite opposite:
the first objective function is unimodal, globally quadratic and ill-conditioned
with a conditioning of about \(10^6\) with smooth local irregularities while
the second objective is highly multimodal with a structured but highly asymmetric
placement of the optima. Constructed as a deceptive function for symmetrically
distributed search operators.

Contained in the \emph{separable - separable} function group.

% #1: function string (2 digits)
% #2: dimension string (2 digits)
% #3: instance string (2 digits)
% #4: function number
% #5: dimension
\bbobfigs{60}{05}{01}{60}{5}


\subsection[\texorpdfstring{\protect\(F_{61}\protect\): Separable Ellipsoid/Linear Slope}{F61: Separable Ellipsoid/Linear Slope}]{\texorpdfstring{\protect\(F_{61}\protect\): Separable Ellipsoid/Linear Slope}{}}
\label{index:f61}\label{index:separable-ellipsoid-linear-slope}
Combination of the separable Ellipsoid function (\href{https://coco.gforge.inria.fr/downloads/download16.00/bbobdocfunctions.pdf\#page=10}{\(f_2\) in the \bbob suite}) and the
Linear Slope function (\href{https://coco.gforge.inria.fr/downloads/download16.00/bbobdocfunctions.pdf\#page=25}{\(f_5\) in the \bbob suite}).

Both objective functions are separable. The first objective function
is unimodal with a high condition number of about \(10^6\). The second
objective function is fully linear within the hypercube \([-5,5]^n\) and
has a region of constant \(f\)-value outside the hypercube by definition to ensure
that a solution at one corner of \([-5,5]^n\) has optimal function value.

Contained in the \emph{separable - separable} function group.

% #1: function string (2 digits)
% #2: dimension string (2 digits)
% #3: instance string (2 digits)
% #4: function number
% #5: dimension
\bbobfigs{61}{05}{01}{61}{5}


\subsection[\texorpdfstring{\protect\(F_{62}\protect\): separable Rastrigin/B\"{u}che-Rastrigin}{F62: separable Rastrigin/B\"{u}che-Rastrigin}]{\texorpdfstring{\protect\(F_{62}\protect\): separable Rastrigin/B\"{u}che-Rastrigin}{}}
\label{index:separable-rastrigin-buche-rastrigin}\label{index:f62}
Combination of the separable Rastrigin function (\href{https://coco.gforge.inria.fr/downloads/download16.00/bbobdocfunctions.pdf\#page=15}{\(f_3\) in the \bbob suite}) and the
separable B\"{u}che-Rastrigin function (\href{https://coco.gforge.inria.fr/downloads/download16.00/bbobdocfunctions.pdf\#page=20}{\(f_4\) in the \bbob suite}).

Both objective functions are separable and highly multimodal with an underlying
structure for the placements of the optima. While for the separable Rastrigin function,
the placements of the optima is symmetric, the optima for the B\"{u}che-Rastrigin function
are highly asymmetrically placed.

Contained in the \emph{separable - separable} function group.

% #1: function string (2 digits)
% #2: dimension string (2 digits)
% #3: instance string (2 digits)
% #4: function number
% #5: dimension
\bbobfigs{62}{05}{01}{62}{5}


\subsection[\texorpdfstring{\protect\(F_{63}\protect\): Separable Rastrigin/Linear Slope}{F63: Separable Rastrigin/Linear Slope}]{\texorpdfstring{\protect\(F_{63}\protect\): Separable Rastrigin/Linear Slope}{}}
\label{index:separable-rastrigin-linear-slope}\label{index:f63}
Combination of the separable Rastrigin function (\href{https://coco.gforge.inria.fr/downloads/download16.00/bbobdocfunctions.pdf\#page=15}{\(f_3\) in the \bbob suite}) and the
Linear Slope function (\href{https://coco.gforge.inria.fr/downloads/download16.00/bbobdocfunctions.pdf\#page=25}{\(f_5\) in the \bbob suite}).

Both objective functions are separable, but while the first objective function
is highly multimodal with an underlying symmetric structure, the second objective
function is purely linear with plateaus of constant function value outside
the region \([-5,5]^n\).

Contained in the \emph{separable - separable} function group.

% #1: function string (2 digits)
% #2: dimension string (2 digits)
% #3: instance string (2 digits)
% #4: function number
% #5: dimension
\bbobfigs{63}{05}{01}{63}{5}


\subsection[\texorpdfstring{\protect\(F_{64}\protect\): B\"{u}che-Rastrigin/Linear slope}{F64: B\"{u}che-Rastrigin/Linear slope}]{\texorpdfstring{\protect\(F_{64}\protect\): B\"{u}che-Rastrigin/Linear slope}{}}
\label{index:f64}\label{index:buche-rastrigin-linear-slope}
Combination of the B\"{u}che-Rastrigin function (\href{https://coco.gforge.inria.fr/downloads/download16.00/bbobdocfunctions.pdf\#page=20}{\(f_4\) in the \bbob suite}) and the
Linear Slope function (\href{https://coco.gforge.inria.fr/downloads/download16.00/bbobdocfunctions.pdf\#page=25}{\(f_5\) in the \bbob suite}).

Both objective functions are separable, but while the first objective function
is highly multimodal with an underlying asymmetric structure, the second objective
function is purely linear with plateaus of constant function value outside
the region \([-5,5]^n\).

Contained in the \emph{separable - separable} function group.

% #1: function string (2 digits)
% #2: dimension string (2 digits)
% #3: instance string (2 digits)
% #4: function number
% #5: dimension
\bbobfigs{64}{05}{01}{64}{5}


\subsection[\texorpdfstring{\protect\(F_{65}\protect\): Attractive Sector/Step-ellipsoid}{F65: Attractive Sector/Step-ellipsoid}]{\texorpdfstring{\protect\(F_{65}\protect\): Attractive Sector/Step-ellipsoid}{}}
\label{index:f65}\label{index:attractive-sector-step-ellipsoid}
Combination of the Attractive Sector function (\href{https://coco.gforge.inria.fr/downloads/download16.00/bbobdocfunctions.pdf\#page=30}{\(f_6\) in the \bbob suite}) and the
Step Ellipsoidal function (\href{https://coco.gforge.inria.fr/downloads/download16.00/bbobdocfunctions.pdf\#page=35}{\(f_7\) in the \bbob suite}).

Both objective functions are unimodal and of moderate conditioning.
The first objective function is highly asymmetric, where only one \emph{hypercone}
(with angular base area) with a volume of roughly \(1/2^n\) yields low function values.
The optimum of the first objective is located at the tip of this cone. This function can
be deceptive for cumulative step size adaptation. The second objective function consists
of many plateaus of different sizes. Apart from a small area close to the global optimum,
the gradient is zero almost everywhere.

Contained in the \emph{moderate - moderate} function group.

% #1: function string (2 digits)
% #2: dimension string (2 digits)
% #3: instance string (2 digits)
% #4: function number
% #5: dimension
\bbobfigs{65}{05}{01}{65}{5}


\subsection[\texorpdfstring{\protect\(F_{66}\protect\): Attractive Sector/rotated Rosenbrock}{F66: Attractive Sector/rotated Rosenbrock}]{\texorpdfstring{\protect\(F_{66}\protect\): Attractive Sector/rotated Rosenbrock}{}}
\label{index:f66}\label{index:attractive-sector-rotated-rosenbrock}
Combination of the Attractive Sector function (\href{https://coco.gforge.inria.fr/downloads/download16.00/bbobdocfunctions.pdf\#page=30}{\(f_6\) in the \bbob suite}) and the
rotated Rosenbrock function (\href{https://coco.gforge.inria.fr/downloads/download16.00/bbobdocfunctions.pdf\#page=45}{\(f_9\) in the \bbob suite}).

The first objective function is highly asymmetric, where only one \emph{hypercone}
(with angular base area) with a volume of roughly \(1/2^n\) yields low function values.
The optimum of the first objective is located at the tip of this cone.
The second objective function is the so-called banana function due to its 2-D contour lines
as a bent ridge (or valley) and partially separable (tri-band structure). In larger dimensions,
the second objective function has a local optimum with an attraction volume of about 25\%.
Note that, compared to the original Rosenbrock function, a rotation in the search space
is applied, such that the second objective function is non-separable.

Contained in the \emph{moderate - moderate} function group.

% #1: function string (2 digits)
% #2: dimension string (2 digits)
% #3: instance string (2 digits)
% #4: function number
% #5: dimension
\bbobfigs{66}{05}{01}{66}{5}


\subsection[\texorpdfstring{\protect\(F_{67}\protect\): Step-ellipsoid/separable Rosenbrock}{F67: Step-ellipsoid/separable Rosenbrock}]{\texorpdfstring{\protect\(F_{67}\protect\): Step-ellipsoid/separable Rosenbrock}{}}
\label{index:f67}\label{index:step-ellipsoid-separable-rosenbrock}
Combination of the Step Ellipsoidal function (\href{https://coco.gforge.inria.fr/downloads/download16.00/bbobdocfunctions.pdf\#page=35}{\(f_7\) in the \bbob suite}) and the
separable Rosenbrock function (\href{https://coco.gforge.inria.fr/downloads/download16.00/bbobdocfunctions.pdf\#page=40}{\(f_8\) in the \bbob suite}).

The first objective function is unimodal, non-separable, and has a
conditioning of about 100. It actually consists of many plateaus of different sizes.
Apart from a small area close to the global optimum, the gradient is zero almost everywhere.
The second objective function is the so-called banana function due to its 2-D contour lines
as a bent ridge (or valley). It is partially separable (tri-band structure) and
in larger dimensions, the function has a local optimum with an attraction volume of about 25\%.

Contained in the \emph{moderate - moderate} function group.

% #1: function string (2 digits)
% #2: dimension string (2 digits)
% #3: instance string (2 digits)
% #4: function number
% #5: dimension
\bbobfigs{67}{05}{01}{67}{5}


\subsection[\texorpdfstring{\protect\(F_{68}\protect\): Step-ellipsoid/rotated Rosenbrock}{F68: Step-ellipsoid/rotated Rosenbrock}]{\texorpdfstring{\protect\(F_{68}\protect\): Step-ellipsoid/rotated Rosenbrock}{}}
\label{index:step-ellipsoid-rotated-rosenbrock}\label{index:f68}
Combination of the Step Ellipsoidal function (\href{https://coco.gforge.inria.fr/downloads/download16.00/bbobdocfunctions.pdf\#page=35}{\(f_7\) in the \bbob suite}) and the
rotated Rosenbrock function (\href{https://coco.gforge.inria.fr/downloads/download16.00/bbobdocfunctions.pdf\#page=45}{\(f_9\) in the \bbob suite}).

The first objective function is unimodal, non-separable, and has a
conditioning of about 100. It actually consists of many plateaus of different sizes.
Apart from a small area close to the global optimum, the gradient is zero almost everywhere.
The second objective function is a rotated version of the original
so-called banana function (due to its 2-D contour lines as a bent ridge or valley) and
in larger dimensions, has a local optimum with an attraction volume of about 25\%.

This function resembles \(F_{67}\) except for the additional search space
rotation for the second objective function which makes both objective
function fully non-separable.

Contained in the \emph{moderate - moderate} function group.

% #1: function string (2 digits)
% #2: dimension string (2 digits)
% #3: instance string (2 digits)
% #4: function number
% #5: dimension
\bbobfigs{68}{05}{01}{68}{5}


\subsection[\texorpdfstring{\protect\(F_{69}\protect\): separable Rosenbrock/rotated Rosenbrock}{F69: separable Rosenbrock/rotated Rosenbrock}]{\texorpdfstring{\protect\(F_{69}\protect\): separable Rosenbrock/rotated Rosenbrock}{}}
\label{index:separable-rosenbrock-rotated-rosenbrock}\label{index:f69}
Combination of the separable Rosenbrock function (\href{https://coco.gforge.inria.fr/downloads/download16.00/bbobdocfunctions.pdf\#page=40}{\(f_8\) in the \bbob suite}) and the
rotated Rosenbrock function (\href{https://coco.gforge.inria.fr/downloads/download16.00/bbobdocfunctions.pdf\#page=45}{\(f_9\) in the \bbob suite}).

Both objective functions are Rosenbrock functions (also known under the name
banana function due to its 2-D contour lines forming a bent ridge or valley)
with a local optimum in large dimension that has about 25\% attraction volume.
The first objective function is partially separable while the second objective
function is fully non-separable.

Contained in the \emph{moderate - moderate} function group.

% #1: function string (2 digits)
% #2: dimension string (2 digits)
% #3: instance string (2 digits)
% #4: function number
% #5: dimension
\bbobfigs{69}{05}{01}{69}{5}


\subsection[\texorpdfstring{\protect\(F_{70}\protect\): Ellipsoid/Discus}{F70: Ellipsoid/Discus}]{\texorpdfstring{\protect\(F_{70}\protect\): Ellipsoid/Discus}{}}
\label{index:ellipsoid-discus}\label{index:f70}
Combination of the Ellipsoid function (\href{https://coco.gforge.inria.fr/downloads/download16.00/bbobdocfunctions.pdf\#page=50}{\(f_{10}\) in the \bbob suite}) and the
Discus (or Tablet) function (\href{https://coco.gforge.inria.fr/downloads/download16.00/bbobdocfunctions.pdf\#page=55}{\(f_{11}\) in the \bbob suite}).

Both objective functions are globally quadratic (unimodal) ill-conditioned functions with
condition numbers of \(10^6\) with  smooth local irregularities. A single direction
in search space is a thousand times more sensitive than all others for the Discus function.

Contained in the \emph{ill-conditioned - ill-conditioned} function group.

% #1: function string (2 digits)
% #2: dimension string (2 digits)
% #3: instance string (2 digits)
% #4: function number
% #5: dimension
\bbobfigs{70}{05}{01}{70}{5}


\subsection[\texorpdfstring{\protect\(F_{71}\protect\): Ellipsoid/Bent Cigar}{F71: Ellipsoid/Bent Cigar}]{\texorpdfstring{\protect\(F_{71}\protect\): Ellipsoid/Bent Cigar}{}}
\label{index:ellipsoid-bent-cigar}\label{index:f71}
Combination of the Ellipsoid function (\href{https://coco.gforge.inria.fr/downloads/download16.00/bbobdocfunctions.pdf\#page=50}{\(f_{10}\) in the \bbob suite}) and the
Bent Cigar function (\href{https://coco.gforge.inria.fr/downloads/download16.00/bbobdocfunctions.pdf\#page=60}{\(f_{12}\) in the \bbob suite}).

Both objective functions are unimodal, non-separable, and have a conditioning
of about \(10^6\). The Ellipsoid function is globally quadratic with
smooth local irregularities while the Bent Cigar function deviates remarkably
from being quadratic due to an asymmetric transformation. To optimize the
Bent Cigar function, a smooth, but very narrow ridge has to be followed.

Contained in the \emph{ill-conditioned - ill-conditioned} function group.

% #1: function string (2 digits)
% #2: dimension string (2 digits)
% #3: instance string (2 digits)
% #4: function number
% #5: dimension
\bbobfigs{71}{05}{01}{71}{5}


\subsection[\texorpdfstring{\protect\(F_{72}\protect\): Ellipsoid/Sharp Ridge}{F72: Ellipsoid/Sharp Ridge}]{\texorpdfstring{\protect\(F_{72}\protect\): Ellipsoid/Sharp Ridge}{}}
\label{index:ellipsoid-sharp-ridge}\label{index:f72}
Combination of the Ellipsoid function (\href{https://coco.gforge.inria.fr/downloads/download16.00/bbobdocfunctions.pdf\#page=50}{\(f_{10}\) in the \bbob suite}) and the
Sharp Ridge function (\href{https://coco.gforge.inria.fr/downloads/download16.00/bbobdocfunctions.pdf\#page=65}{\(f_{13}\) in the \bbob suite}).

Both objective functions are unimodal, non-separable, and have a conditioning
of about \(10^6\). Compared to the previous function, the ridge of the
here is sharp (non-differentiable) and the gradient remains constant, when the
ridge is approached from a given point. Approaching the ridge is
initially effective, but search behavior becomes difficult to diagnose
close to the ridge because the gradient towards the ridge does not flatten out.

Contained in the \emph{ill-conditioned - ill-conditioned} function group.

% #1: function string (2 digits)
% #2: dimension string (2 digits)
% #3: instance string (2 digits)
% #4: function number
% #5: dimension
\bbobfigs{72}{05}{01}{72}{5}


\subsection[\texorpdfstring{\protect\(F_{73}\protect\): Ellipsoid/Sum of Different Powers}{F73: Ellipsoid/Sum of Different Powers}]{\texorpdfstring{\protect\(F_{73}\protect\): Ellipsoid/Sum of Different Powers}{}}
\label{index:ellipsoid-sum-of-different-powers}\label{index:f73}
Combination of the Ellipsoid function (\href{https://coco.gforge.inria.fr/downloads/download16.00/bbobdocfunctions.pdf\#page=50}{\(f_{10}\) in the \bbob suite}) and the
Sum of Different Powers function (\href{https://coco.gforge.inria.fr/downloads/download16.00/bbobdocfunctions.pdf\#page=70}{\(f_{14}\) in the \bbob suite}).

Both objective functions are unimodal and non-separable. While the
Ellipsoid function has a constant conditioning of \(10^6\) everywhere,
the sensitivities of the \(z_i\)-variables (in the rotated search space)
for the Different Powers function become more and more different when
approaching the optimum. The latter function has furthermore a small
solution volume.

Contained in the \emph{ill-conditioned - ill-conditioned} function group.

% #1: function string (2 digits)
% #2: dimension string (2 digits)
% #3: instance string (2 digits)
% #4: function number
% #5: dimension
\bbobfigs{73}{05}{01}{73}{5}


\subsection[\texorpdfstring{\protect\(F_{74}\protect\): Discus/Bent Cigar}{F74: Discus/Bent Cigar}]{\texorpdfstring{\protect\(F_{74}\protect\): Discus/Bent Cigar}{}}
\label{index:f74}\label{index:discus-bent-cigar}
Combination of the Discus function (\href{https://coco.gforge.inria.fr/downloads/download16.00/bbobdocfunctions.pdf\#page=55}{\(f_{11}\) in the \bbob suite}) and the
Bent cigar function (\href{https://coco.gforge.inria.fr/downloads/download16.00/bbobdocfunctions.pdf\#page=60}{\(f_{12}\) in the \bbob suite}).

Both objective functions are unimodal, non-separable, and have a conditioning
of about \(10^6\).
The Discus function is globally quadratic with smooth local irregularities
and has a single direction in search space that is a thousand times more
sensitive than all others. The Bent Cigar function deviates remarkably
from being quadratic due to an asymmetric transformation and a smooth,
but very narrow ridge has to be followed to optimize it.

Contained in the \emph{ill-conditioned - ill-conditioned} function group.

% #1: function string (2 digits)
% #2: dimension string (2 digits)
% #3: instance string (2 digits)
% #4: function number
% #5: dimension
\bbobfigs{74}{05}{01}{74}{5}


\subsection[\texorpdfstring{\protect\(F_{75}\protect\): Discus/Sharp Ridge}{F75: Discus/Sharp Ridge}]{\texorpdfstring{\protect\(F_{75}\protect\): Discus/Sharp Ridge}{}}
\label{index:discus-sharp-ridge}\label{index:f75}
Combination of the Discus function (\href{https://coco.gforge.inria.fr/downloads/download16.00/bbobdocfunctions.pdf\#page=55}{\(f_{11}\) in the \bbob suite}) and the
Sharp Ridge function (\href{https://coco.gforge.inria.fr/downloads/download16.00/bbobdocfunctions.pdf\#page=65}{\(f_{13}\) in the \bbob suite}).

Both objective functions are unimodal, non-separable, and have a conditioning
of about \(10^6\).
The Discus function is globally quadratic with smooth local irregularities
and has a single direction in search space that is a thousand times more
sensitive than all others.
To optimize the Sharp Ridge function, a sharp (i.e. non-differentiable) ridge
has to be followed around which the gradient remains constant, when the
ridge is approached from a given point. Approaching the ridge is
initially effective, but search behavior becomes difficult to diagnose
close to the ridge because the gradient towards the ridge does not flatten out.

Contained in the \emph{ill-conditioned - ill-conditioned} function group.

% #1: function string (2 digits)
% #2: dimension string (2 digits)
% #3: instance string (2 digits)
% #4: function number
% #5: dimension
\bbobfigs{75}{05}{01}{75}{5}


\subsection[\texorpdfstring{\protect\(F_{76}\protect\): Discus/Sum of Different Powers}{F76: Discus/Sum of Different Powers}]{\texorpdfstring{\protect\(F_{76}\protect\): Discus/Sum of Different Powers}{}}
\label{index:f76}\label{index:discus-sum-of-different-powers}
Combination of the Discus function (\href{https://coco.gforge.inria.fr/downloads/download16.00/bbobdocfunctions.pdf\#page=55}{\(f_{11}\) in the \bbob suite}) and the
Sum of Different Powers function (\href{https://coco.gforge.inria.fr/downloads/download16.00/bbobdocfunctions.pdf\#page=70}{\(f_{14}\) in the \bbob suite}).

Both objective functions are unimodal and non-separable. While the
globally quadratic Discus function has a constant conditioning
of about \(10^6\) everywhere
with a single direction in search space that is a thousand times more
sensitive than all others, the sensitivities of the
\(z_i\)-variables (in the rotated search space)
for the Different Powers function become more and more different when
approaching the optimum. The latter function has furthermore a small
solution volume.

Contained in the \emph{ill-conditioned - ill-conditioned} function group.

% #1: function string (2 digits)
% #2: dimension string (2 digits)
% #3: instance string (2 digits)
% #4: function number
% #5: dimension
\bbobfigs{76}{05}{01}{76}{5}


\subsection[\texorpdfstring{\protect\(F_{77}\protect\): Bent Cigar/Sharp Ridge}{F77: Bent Cigar/Sharp Ridge}]{\texorpdfstring{\protect\(F_{77}\protect\): Bent Cigar/Sharp Ridge}{}}
\label{index:bent-cigar-sharp-ridge}\label{index:f77}
Combination of the Bent Cigar function (\href{https://coco.gforge.inria.fr/downloads/download16.00/bbobdocfunctions.pdf\#page=60}{\(f_{12}\) in the \bbob suite}) and the
Sharp Ridge function (\href{https://coco.gforge.inria.fr/downloads/download16.00/bbobdocfunctions.pdf\#page=65}{\(f_{13}\) in the \bbob suite}).

Both objective functions are unimodal, non-separable, and have a conditioning
of about \(10^6\).
The Bent Cigar function deviates remarkably
from being quadratic due to an asymmetric transformation and a smooth,
but very narrow ridge has to be followed to optimize it.
To optimize the Sharp Ridge function, in turn, the ridge to be followed
is even sharper (i.e. non-differentiable), around which the gradient remains
constant, when the ridge is approached from a given point. Approaching the ridge is
initially effective, but search behavior becomes difficult to diagnose
close to the ridge because the gradient towards the ridge does not flatten out.

Contained in the \emph{ill-conditioned - ill-conditioned} function group.

% #1: function string (2 digits)
% #2: dimension string (2 digits)
% #3: instance string (2 digits)
% #4: function number
% #5: dimension
\bbobfigs{77}{05}{01}{77}{5}


\subsection[\texorpdfstring{\protect\(F_{78}\protect\): Bent Cigar/Sum of Different Powers}{F78: Bent Cigar/Sum of Different Powers}]{\texorpdfstring{\protect\(F_{78}\protect\): Bent Cigar/Sum of Different Powers}{}}
\label{index:f78}\label{index:bent-cigar-sum-of-different-powers}
Combination of the Bent Cigar function (\href{https://coco.gforge.inria.fr/downloads/download16.00/bbobdocfunctions.pdf\#page=60}{\(f_{12}\) in the \bbob suite}) and the
Sum of Different Powers function (\href{https://coco.gforge.inria.fr/downloads/download16.00/bbobdocfunctions.pdf\#page=70}{\(f_{14}\) in the \bbob suite}).

Both objective functions are unimodal, non-separable, and have a conditioning
of about \(10^6\).

Both objective functions are unimodal and non-separable but differ in the
difficulties provided to an optimization algorithm.
The Bent Cigar function, on the one hand, deviates remarkably
from being quadratic due to an asymmetric transformation and a smooth,
but very narrow ridge has to be followed to optimize it.
The sensitivities of the
\(z_i\)-variables (in the rotated search space)
for the Different Powers function, on the other hand,
become more and more different when
approaching the optimum.

Contained in the \emph{ill-conditioned - ill-conditioned} function group.

% #1: function string (2 digits)
% #2: dimension string (2 digits)
% #3: instance string (2 digits)
% #4: function number
% #5: dimension
\bbobfigs{78}{05}{01}{78}{5}


\subsection[\texorpdfstring{\protect\(F_{79}\protect\): Rastrigin/Schaffer F7 with conditioning of 1000}{F79: Rastrigin/Schaffer F7 with conditioning of 1000}]{\texorpdfstring{\protect\(F_{79}\protect\): Rastrigin/Schaffer F7 with conditioning of 1000}{}}
\label{index:rastrigin-schaffer-f7-with-conditioning-of-1000}\label{index:f79}
Combination of the Rastrigin function (\href{https://coco.gforge.inria.fr/downloads/download16.00/bbobdocfunctions.pdf\#page=75}{\(f_{15}\) in the \bbob suite}) and the
Schaffer F7 function with conditioning 1000 (\href{https://coco.gforge.inria.fr/downloads/download16.00/bbobdocfunctions.pdf\#page=90}{\(f_{18}\) in the \bbob suite}).

Both objective functions are non-separable and highly multimodal.
The problem's Rastrigin function alleviates the symmetry and regularity of
the originally proposed Rastrigin function via asymmetric and oscillating
transformations of the search space. It has roughly \(10^n\) local optima,
a low conditioning, and the global amplitude of function values is large
compared to the local amplitudes.
On the contrary, frequency  and  amplitude  of  the  function value modulation
vary for the Schaffer F7 function. It is furthermore asymmetric as well
but, compared to the other objective function is moderately ill-conditioned
with a conditioning of 1000.

Contained in the \emph{multimodal - multimodal} function group.

% #1: function string (2 digits)
% #2: dimension string (2 digits)
% #3: instance string (2 digits)
% #4: function number
% #5: dimension
\bbobfigs{79}{05}{01}{79}{5}


\subsection[\texorpdfstring{\protect\(F_{80}\protect\): Rastrigin/Griewank-Rosenbrock}{F80: Rastrigin/Griewank-Rosenbrock}]{\texorpdfstring{\protect\(F_{80}\protect\): Rastrigin/Griewank-Rosenbrock}{}}
\label{index:rastrigin-griewank-rosenbrock}\label{index:f80}
Combination of the Rastrigin function (\href{https://coco.gforge.inria.fr/downloads/download16.00/bbobdocfunctions.pdf\#page=75}{\(f_{15}\) in the \bbob suite}) and the
Griewank-Rosenbrock function (\href{https://coco.gforge.inria.fr/downloads/download16.00/bbobdocfunctions.pdf\#page=95}{\(f_{19}\) in the \bbob suite}).

Both objective functions are non-separable and highly multimodal.
Both objective functions furthermore are variants of the original Rosenbrock
function:
The problem's Rastrigin function alleviates the symmetry and regularity of
the originally proposed Rastrigin function via asymmetric and oscillating
transformations of the search space. The Griewank-Rosenbrock function
resembles the original Rosenbrock function in a highly multimodal way.

Contained in the \emph{multimodal multimodal} function group.

% #1: function string (2 digits)
% #2: dimension string (2 digits)
% #3: instance string (2 digits)
% #4: function number
% #5: dimension
\bbobfigs{80}{05}{01}{80}{5}


\subsection[\texorpdfstring{\protect\(F_{81}\protect\): Schaffer F7/Schaffer F7 with conditioning 1000}{F81: Schaffer F7/Schaffer F7 with conditioning 1000}]{\texorpdfstring{\protect\(F_{81}\protect\): Schaffer F7/Schaffer F7 with conditioning 1000}{}}
\label{index:schaffer-f7-schaffer-f7-with-conditioning-1000}\label{index:f81}
Combination of the Schaffer F7 function (\href{https://coco.gforge.inria.fr/downloads/download16.00/bbobdocfunctions.pdf\#page=85}{\(f_{17}\) in the \bbob suite}) and the
Schaffer F7 with conditioning 1000 function (\href{https://coco.gforge.inria.fr/downloads/download16.00/bbobdocfunctions.pdf\#page=90}{\(f_{18}\) in the \bbob suite}).

Both objective functions are of the same type (asymmetric, non-separable,
highly multimodal where frequency and amplitude of the modulation vary).
The main difference is in the conditioning, which is about 10 in one case
and 1000 in the other.

Contained in the \emph{multimodal - multimodal} function group.

% #1: function string (2 digits)
% #2: dimension string (2 digits)
% #3: instance string (2 digits)
% #4: function number
% #5: dimension
\bbobfigs{81}{05}{01}{81}{5}


\subsection[\texorpdfstring{\protect\(F_{82}\protect\): Schaffer F7/Griewank-Rosenbrock}{F82: Schaffer F7/Griewank-Rosenbrock}]{\texorpdfstring{\protect\(F_{82}\protect\): Schaffer F7/Griewank-Rosenbrock}{}}
\label{index:schaffer-f7-griewank-rosenbrock}\label{index:f82}
Combination of the Schaffer F7 function (\href{https://coco.gforge.inria.fr/downloads/download16.00/bbobdocfunctions.pdf\#page=85}{\(f_{17}\) in the \bbob suite}) and the
Griewank-Rosenbrock function (\href{https://coco.gforge.inria.fr/downloads/download16.00/bbobdocfunctions.pdf\#page=95}{\(f_{19}\) in the \bbob suite}).

Both objective functions are non-separable and highly multimodal.
For the asymmetric Schaffer F7 function, frequency and amplitude of
the function value modulation vary and it has a low conditioning of
about 10. The Griewank-Rosenbrock function
resembles the original Rosenbrock function in a highly multimodal way.

Contained in the \emph{multimodal - multimodal} function group.

% #1: function string (2 digits)
% #2: dimension string (2 digits)
% #3: instance string (2 digits)
% #4: function number
% #5: dimension
\bbobfigs{82}{05}{01}{82}{5}


\subsection[\texorpdfstring{\protect\(F_{83}\protect\): Schaffer F7 with conditioning 1000/Griewank-Rosenbrock}{F83: Schaffer F7 with conditioning 1000/Griewank-Rosenbrock}]{\texorpdfstring{\protect\(F_{83}\protect\): Schaffer F7 with conditioning 1000/Griewank-Rosenbrock}{}}
\label{index:schaffer-f7-with-conditioning-1000-griewank-rosenbrock}\label{index:f83}
Combination of the Schaffer F7 function with conditioning 1000 (\href{https://coco.gforge.inria.fr/downloads/download16.00/bbobdocfunctions.pdf\#page=90}{\(f_{18}\) in the \bbob suite}) and the
Griewank-Rosenbrock function (\href{https://coco.gforge.inria.fr/downloads/download16.00/bbobdocfunctions.pdf\#page=95}{\(f_{19}\) in the \bbob suite}).

Compared to \(F_{82}\), the only difference is the higher condition number of
about 1000 (compared to 10) in the Schaffer F7 function.

Contained in the \emph{multimodal - multimodal} function group.

% #1: function string (2 digits)
% #2: dimension string (2 digits)
% #3: instance string (2 digits)
% #4: function number
% #5: dimension
\bbobfigs{83}{05}{01}{83}{5}


\subsection[\texorpdfstring{\protect\(F_{84}\protect\): Schwefel/Gallagher 21}{F84: Schwefel/Gallagher 21}]{\texorpdfstring{\protect\(F_{84}\protect\): Schwefel/Gallagher 21}{}}
\label{index:f84}\label{index:schwefel-gallagher-21}
Combination of the Schwefel function (\href{https://coco.gforge.inria.fr/downloads/download16.00/bbobdocfunctions.pdf\#page=100}{\(f_{20}\) in the \bbob suite}) and the
Gallagher 21 function (\href{https://coco.gforge.inria.fr/downloads/download16.00/bbobdocfunctions.pdf\#page=110}{\(f_{22}\) in the \bbob suite}).

Both objective functions are multimodal with only a weak global structure.
The most prominent \(2^n\) minima of the Schwefel function
are located comparatively close to the corners of
the unpenalized search area. The penalization is essential, as otherwise more
and better minima occur further away from the search space origin. The function
is furthermore partially separable, a kind of combinatorial problem, and has two
search regimes. The Gallagher function consists of 21 optima with position and
height being unrelated and randomly chosen (different for each instantiation
of the function). The conditioning of the Gallagher function around the global
optimum is about 1000.

Contained in the \emph{weakly-structured - weakly-structured} function group.

% #1: function string (2 digits)
% #2: dimension string (2 digits)
% #3: instance string (2 digits)
% #4: function number
% #5: dimension
\bbobfigs{84}{05}{01}{84}{5}


\subsection[\texorpdfstring{\protect\(F_{85}\protect\): Schwefel/Katsuuras}{F85: Schwefel/Katsuuras}]{\texorpdfstring{\protect\(F_{85}\protect\): Schwefel/Katsuuras}{}}
\label{index:f85}\label{index:schwefel-katsuuras}
Combination of the Schwefel function (\href{https://coco.gforge.inria.fr/downloads/download16.00/bbobdocfunctions.pdf\#page=100}{\(f_{20}\) in the \bbob suite}) and the
Katsuuras function (\href{https://coco.gforge.inria.fr/downloads/download16.00/bbobdocfunctions.pdf\#page=115}{\(f_{23}\) in the \bbob suite}).

Both objective functions are highly multimodal with an exponential number
(in the dimension) of (global) optima and only a weak global structure.
The most prominent \(2^n\) minima of the Schwefel function
are located comparatively close to the corners of
the unpenalized search area.
The Katsuuras function, in turn, is highly repetitive with more than \(10^n\)
global optima.

% #1: function string (2 digits)
% #2: dimension string (2 digits)
% #3: instance string (2 digits)
% #4: function number
% #5: dimension
\bbobfigs{85}{05}{01}{85}{5}


\subsection[\texorpdfstring{\protect\(F_{86}\protect\): Schwefel/Lunacek bi-Rastrigin}{F86: Schwefel/Lunacek bi-Rastrigin}]{\texorpdfstring{\protect\(F_{86}\protect\): Schwefel/Lunacek bi-Rastrigin}{}}
\label{index:f86}\label{index:schwefel-lunacek-bi-rastrigin}
Combination of the Schwefel function (\href{https://coco.gforge.inria.fr/downloads/download16.00/bbobdocfunctions.pdf\#page=100}{\(f_{20}\) in the \bbob suite}) and the
Lunacek bi-Rastrigin function (\href{https://coco.gforge.inria.fr/downloads/download16.00/bbobdocfunctions.pdf\#page=120}{\(f_{24}\) in the \bbob suite}).

Both objective functions are highly multimodal with only a weak global structure.
While the most prominent \(2^n\) minima of the Schwefel function
are located comparatively close to the corners of the unpenalized search area,
the Lunacek bi-Rastrigin function has two superimposed funnels. Presumably,
different approaches need to be used for ``selecting the funnel''
and for searching the highly multimodal function ``within'' the funnel.
The single-objective Lunacek bi-Rastrigin function was constructed
to be deceptive for some evolutionary algorithms with large population size.

Contained in the \emph{weakly-structure - weakly-structured} function group.

% #1: function string (2 digits)
% #2: dimension string (2 digits)
% #3: instance string (2 digits)
% #4: function number
% #5: dimension
\bbobfigs{86}{05}{01}{86}{5}


\subsection[\texorpdfstring{\protect\(F_{87}\protect\): Gallagher 101/Gallagher 21}{F87: Gallagher 101/Gallagher 21}]{\texorpdfstring{\protect\(F_{87}\protect\): Gallagher 101/Gallagher 21}{}}
\label{index:f87}\label{index:gallagher-101-gallagher-21}
Combination of Gallagher’s Gaussian 101-me Peaks function (\href{https://coco.gforge.inria.fr/downloads/download16.00/bbobdocfunctions.pdf\#page=105}{\(f_{21}\) in the \bbob suite}) and the
Gallagher’s Gaussian 21-hi Peaks function (\href{https://coco.gforge.inria.fr/downloads/download16.00/bbobdocfunctions.pdf\#page=110}{\(f_{22}\) in the \bbob suite}).

Both objective functions are multimodal and non-separable. Both consist of a set of
optima with position and height being unrelated and randomly chosen. The number of
optima is 101 and 21 respectively and the condition number around the (unique) global
optima are about 30 and about 1000 respectively.

Contained in the \emph{weakly-structured - weakly-structured} function group.

% #1: function string (2 digits)
% #2: dimension string (2 digits)
% #3: instance string (2 digits)
% #4: function number
% #5: dimension
\bbobfigs{87}{05}{01}{87}{5}


\subsection[\texorpdfstring{\protect\(F_{88}\protect\): Gallagher 101/Katsuuras}{F88: Gallagher 101/Katsuuras}]{\texorpdfstring{\protect\(F_{88}\protect\): Gallagher 101/Katsuuras}{}}
\label{index:f88}\label{index:gallagher-101-katsuuras}
Combination of Gallagher’s Gaussian 101-me Peaks function (\href{https://coco.gforge.inria.fr/downloads/download16.00/bbobdocfunctions.pdf\#page=105}{\(f_{21}\) in the \bbob suite}) and the
Katsuuras function (\href{https://coco.gforge.inria.fr/downloads/download16.00/bbobdocfunctions.pdf\#page=115}{\(f_{23}\) in the \bbob suite}).

Both objective functions are non-separable and highly multimodal with
only a weak global structure. Gallagher's Gaussian 101-me Peaks
function consists of a set of 101 optima with position and height being
unrelated and randomly chosen. The conditioning is about 30.
The Katsuuras function, in turn, is highly repetitive with more than \(10^n\)
global optima.

Contained in the \emph{weakly-structured - weakly-structured} function group.

% #1: function string (2 digits)
% #2: dimension string (2 digits)
% #3: instance string (2 digits)
% #4: function number
% #5: dimension
\bbobfigs{88}{05}{01}{88}{5}


\subsection[\texorpdfstring{\protect\(F_{89}\protect\): Gallagher 101/Lunacek bi-Rastrigin}{F89: Gallagher 101/Lunacek bi-Rastrigin}]{\texorpdfstring{\protect\(F_{89}\protect\): Gallagher 101/Lunacek bi-Rastrigin}{}}
\label{index:gallagher-101-lunacek-bi-rastrigin}\label{index:f89}
Combination of Gallagher’s Gaussian 101-me Peaks function (\href{https://coco.gforge.inria.fr/downloads/download16.00/bbobdocfunctions.pdf\#page=105}{\(f_{21}\) in the \bbob suite}) and the
Lunacek bi-Rastrigin function (\href{https://coco.gforge.inria.fr/downloads/download16.00/bbobdocfunctions.pdf\#page=120}{\(f_{24}\) in the \bbob suite}).

Both objective functions are non-separable and highly multimodal with
only a weak global structure. Gallagher's Gaussian 101-me Peaks
function consists of a set of 101 optima with position and height being
unrelated and randomly chosen. The conditioning is about 30.
The Lunacek bi-Rastrigin function has two superimposed funnels. Presumably,
different approaches need to be used for ``selecting the funnel''
and for searching the highly multimodal function ``within'' the funnel.
The single-objective Lunacek bi-Rastrigin function was constructed
to be deceptive for some evolutionary algorithms with large population size.

Contained in the \emph{weakly-structured - weakly-structured} function group.

% #1: function string (2 digits)
% #2: dimension string (2 digits)
% #3: instance string (2 digits)
% #4: function number
% #5: dimension
\bbobfigs{89}{05}{01}{89}{5}


\subsection[\texorpdfstring{\protect\(F_{90}\protect\): Gallagher 21/Katsuuras}{F90: Gallagher 21/Katsuuras}]{\texorpdfstring{\protect\(F_{90}\protect\): Gallagher 21/Katsuuras}{}}
\label{index:gallagher-21-katsuuras}\label{index:f90}
Combination of Gallagher’s Gaussian 21-hi Peaks function (\href{https://coco.gforge.inria.fr/downloads/download16.00/bbobdocfunctions.pdf\#page=110}{\(f_{22}\) in the \bbob suite}) and the
Katsuuras function (\href{https://coco.gforge.inria.fr/downloads/download16.00/bbobdocfunctions.pdf\#page=115}{\(f_{23}\) in the \bbob suite}).

Both objective functions are non-separable and multimodal with
only a weak global structure. Gallagher's Gaussian 21-hi Peaks
function consists of a set of 21 optima with position and height being
unrelated and randomly chosen. The conditioning is about 1000.
The Katsuuras function, in turn, is highly repetitive with more than \(10^n\)
global optima.

Contained in the \emph{weakly-structured - weakly-structured} function group.

% #1: function string (2 digits)
% #2: dimension string (2 digits)
% #3: instance string (2 digits)
% #4: function number
% #5: dimension
\bbobfigs{90}{05}{01}{90}{5}


\subsection[\texorpdfstring{\protect\(F_{91}\protect\): Gallagher 21/Lunacek bi-Rastrigin}{F91: Gallagher 21/Lunacek bi-Rastrigin}]{\texorpdfstring{\protect\(F_{91}\protect\): Gallagher 21/Lunacek bi-Rastrigin}{}}
\label{index:f91}\label{index:gallagher-21-lunacek-bi-rastrigin}
Combination of Gallagher’s Gaussian 21-hi Peaks function (\href{https://coco.gforge.inria.fr/downloads/download16.00/bbobdocfunctions.pdf\#page=110}{\(f_{22}\) in the \bbob suite}) and the
Lunacek bi-Rastrigin function (\href{https://coco.gforge.inria.fr/downloads/download16.00/bbobdocfunctions.pdf\#page=120}{\(f_{24}\) in the \bbob suite}).

Both objective functions are non-separable and multimodal with
only a weak global structure. Gallagher's Gaussian 21-hi Peaks
function consists of a set of 21 optima with position and height being
unrelated and randomly chosen. The conditioning is about 1000.
The Lunacek bi-Rastrigin function has two superimposed funnels. Presumably,
different approaches need to be used for ``selecting the funnel''
and for searching the highly multimodal function ``within'' the funnel.
The single-objective Lunacek bi-Rastrigin function was constructed
to be deceptive for some evolutionary algorithms with large population size.

Contained in the \emph{weakly-structured - weakly-structured} function group.

% #1: function string (2 digits)
% #2: dimension string (2 digits)
% #3: instance string (2 digits)
% #4: function number
% #5: dimension
\bbobfigs{91}{05}{01}{91}{5}


\subsection[\texorpdfstring{\protect\(F_{92}\protect\): Katsuuras/Lunacek bi-Rastrigin}{F92: Katsuuras/Lunacek bi-Rastrigin}]{\texorpdfstring{\protect\(F_{92}\protect\): Katsuuras/Lunacek bi-Rastrigin}{}}
\label{index:f92}\label{index:katsuuras-lunacek-bi-rastrigin}
Combination of the Katsuuras function (\href{https://coco.gforge.inria.fr/downloads/download16.00/bbobdocfunctions.pdf\#page=115}{\(f_{23}\) in the \bbob suite}) and the
Lunacek bi-Rastrigin function (\href{https://coco.gforge.inria.fr/downloads/download16.00/bbobdocfunctions.pdf\#page=120}{\(f_{24}\) in the \bbob suite}).

Both objective functions are non-separable and highly multimodal with
only a weak global structure.
The Katsuuras function is highly repetitive with more than \(10^n\)
global optima.
The Lunacek bi-Rastrigin function has two superimposed funnels. Presumably,
different approaches need to be used for ``selecting the funnel''
and for searching the highly multimodal function ``within'' the funnel.
The single-objective Lunacek bi-Rastrigin function was constructed
to be deceptive for some evolutionary algorithms with large population size.

Contained in the \emph{weakly-structured - weakly-structured} function group.

% #1: function string (2 digits)
% #2: dimension string (2 digits)
% #3: instance string (2 digits)
% #4: function number
% #5: dimension
\bbobfigs{92}{05}{01}{92}{5}

